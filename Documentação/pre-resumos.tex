% ---
% RESUMOS
% ---

% resumo em português
\setlength{\absparsep}{18pt} % ajusta o espaçamento dos parágrafos do resumo
\begin{resumo}
    
    Segundo dados da ABINPET (2021)(LINK AQUI), o Brasil ocupa a sétima posição no Ranking(LINK AQUI) mundial de gastos com animais de estimação. Além disso, no ano de 2019 existiam cerca de 144 milhões de animais domésticos, em sua maioria cães e gatos, e cerca de vinte e cinco por cento do faturamento do ramo Pet(LINK AQUI) é com serviços de Pet Care e Pet Vet.  Tendo em vista que uma das necessidades dos tutores de Pets(LINK AQUI) é o transporte desses seres para consultas, exames, creches caninas, e passeios, etc.O objetivo deste projeto é criar um aplicativo gratuito, de transporte de animais, disponível para celulares Android e IOS pois no mercado nacional não encontramos muitas opções de transporte de animais. Para a criação da aplicação foi realizada uma análise de concorrência, definição de público alvo, verificação de projetos similares e norteadores na literatura, delimitação de tecnologias e linguagens de programação para o desenvolvimento,  estudo de métodos para viabilidade financeira e   captação de receitas. Além disso, no escopo do projeto foram definidas as provas de conceito, produto  mínimo viável, arquitetura da aplicação, banco de dados,   diagrama de classes, diagrama de sequência, critério de segurança, Privacidade e legislação, critérios de log, code convertions, processo de integração contínua, e manutenibilidade da aplicação. Após análises, os autores concluíram que o projeto é totalmente viável e rentável. Para o futuro, espera-se que haja melhorias conforme a necessidade, de modo que o App se torne mais atrativo para os usuários e futuras parcerias.

 \textbf{Palavras-chaves}: latex. abntex. editoração de texto.
\end{resumo}

% resumo em inglês
\begin{resumo}[Abstract]
 \begin{otherlanguage*}{english}

   This is the english abstract.

   \vspace{\onelineskip}

   \noindent 
   \textbf{Keywords}: latex. abntex. text editoration.
 \end{otherlanguage*}
\end{resumo}