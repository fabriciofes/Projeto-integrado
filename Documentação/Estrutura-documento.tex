\chapter{Gerenciamento do Projeto}
Neste capítulo é apresentado o gerenciamento do projeto e todas as etapas seguidas, dentre estas destacam-se as reuniões e Sprint.
\\

\section{Metodologia}
A metodologia de gerenciamento do projeto utilizada é o Scrum, um Framework que tem como objetivo viabilizar o gerenciamento ágil de projetos de software, através de práticas e técnicas que garantem a comunicação efetiva dos integrantes, facilitando o aprendizado e a melhoria contínua das pessoas envolvidas no projeto. \\
Seguindo esse \textit{Framework(LINK AQUI)}, o projeto possui um conjunto de atividades que devem ser executadas em iterações semanais, as quais são chamadas de \textit{Sprint(LINK AQUI)}. De início, são criadas atividades e funcionalidades para o projeto, que são chamadas de \textit{Product Backlog}. Ao iniciar uma \textit{Sprint(LINK AQUI)}, é realizado uma reunião com a equipe para definir quais funcionalidades devem ser implementadas do \textit{Product Backlog}, essa reunião é conhecida como\textit{ Sprint Planning}, as tarefas selecionadas pela equipe são chamadas de \textit{Sprint Backlog} e as funcionalidades escolhidas são definidas como \textit{Users Stories}. Ao final da Sprint(LINK AQUI), a equipe apresenta as funcionalidades implementadas em outra reunião chamada \textit{Sprint Review}, e após isso, o ciclo se reinicia.
\\

\section{Organização da equipe}
A equipe é composta por 5 membros que foram definidos no primeiro dia de aula, onde houve a apresentação da disciplina. Com a equipe formada foi decidido o papel de cada pessoa dentro do projeto de acordo com a experiência de cada um. \\
O Quadro 2 apresenta o papel de cada pessoa no projeto. Vale ressaltar que os integrantes não farão atividades apenas voltadas ao que está no quadro, mas que darão um foco maior nas atividades descritas.
\begin{quadro}[thb]
\centering
\ABNTEXfontereduzida
\caption{Organização da equipe}
\label{quadro-poluido-limpo-desalinhado}
\begin{tabular}{|l|l|l|l|l|l|}
\hline
\thead{Atividades} & \thead{Davi} & \thead{Fabricio} & \thead{José} & \thead{Lorena} & \thead{Guilherme}\\
\hline
Arquitetura& & & \circlemark & & \circlemark \\
\hline
Banco de dados& & \circlemark & \circlemark & & \circlemark \\
\hline
Blog& \circlemark & & & \circlemark & \\
\hline
Código Latex& & \circlemark & & & \\
\hline
Desenvolvimento back-end& & \circlemark & \circlemark & & \circlemark \\
\hline
Desenvolvimento front-end& \circlemark & & & \circlemark & \\
\hline
Documentação& \circlemark & \circlemark & \circlemark & \circlemark & \circlemark \\
\hline
Gerência& & & & \circlemark & \\
\hline
Youtube& \circlemark & & & & \\
\hline
\end{tabular}
\caption{Organização da equipe}
\fonte{Os autores.}
\end{quadro}

O Davi é responsável por realizar as atualizações do blog(LINK AQUI) semanalmente e das apresentações no Youtube e também fará parte do desenvolvimento front-end do projeto.\\ 
O Fabrício é responsável pelo banco de dados e diagramas e é o líder na área de LATEX o trabalho dele será direcionar, ensinar e revisar os códigos LATEX da documentação. \\
O José está encarregado de cuidar do banco de dados juntamente com Fabrício além de ser responsável pelo design de arquitetura e preparar o ambiente de desenvolvimento na primeira etapa do projeto. No decorrer do desenvolvimento ele fará a programação back-end juntamente com o Fabrício.  \\
A Lorena é a gerente de projetos e tem como atividade liderar, organizar, estimular a equipe a sempre trabalharem juntos e compartilhar das dificuldades. Atua no desenvolvimento front-end e realiza as atualizações do blog juntamente com o Davi semanalmente. \\
O Guilherme é responsável pelo design da aplicação, paleta de cores, warframe e suporte na parte de back-end(LINK AQUI) do projeto.\\
Todos os integrantes são responsáveis pelo desenvolvimento da documentação e pela realização dos testes unitários.
\\

\section{Comunicação da equipe}
A equipe disponibiliza vários links com os recursos aqui produzidos, nesses links encontram-se informações sobre o trabalho, documentos, relatos semanais e vídeos de desenvolvimento do projeto(Reuniões e apresentações).  \\
	A \ref{QRYoutube} indica o link para o Youtube utilizado pela equipe para compartilhamento de reuniões, apresentações e etc.\\
Os links das \ref{QRTrello} é referente ao Trello, ferramenta de gerenciamento utilizado pela equipe no início do projeto, nesses sites é possível encontrar o planejamento do projeto.\\
A \ref{QRSVN} indica o link para o repositório Subversion utilizado pela equipe para compartilhamento de arquivos e entrega de documentos.\\
Como requisito da disciplina, o projeto deve ser relatado em um blog, disponível em \ref{QRBlog}.\\
Como objeto adicional para o projeto foi criado um wireframe das telas da aplicação contextualizando, disponível em \ref{QRWireframe}.\\

\newpage
\begin{figure}
    \centering
    \includegraphics{exemplos/QRCode/QR blog.PNG}
    \caption{QR Code - Blog}
    \underline{https://carrarapets.blogspot.com/2022/04/reuniao-com-o-professor-na-terca.html?zx=8b6f737ec7e5c039}
    \label{QRBlog}
\end{figure}
    
\begin{figure}
    \centering
    \includegraphics{exemplos/QRCode/QR SVN.PNG}
    \caption{QR Code - SVN}
    \underline{https://svn.spo.ifsp.edu.br/svn/a6pgp/S202201-PI-NOT/Grupo4/}
    \label{QRSVN}
\end{figure}

\begin{figure}
    \centering
    \includegraphics{exemplos/QRCode/QR trello.PNG}
    \caption{QR Code - Trello}
    \underline{https://trello.com/invite/b/VwkOBdgD/8de44bebb8430adfd8af3871dabc7d4c/projeto-pi1}
    \label{QRTrello}
\end{figure}
    
\begin{figure}
    \centering
    \includegraphics{exemplos/QRCode/QR wireframe.PNG}
    \caption{QR Code - Wireframe}
    \underline{https://whimsical.com/wireframe-tcc-Hi9cvRoYkUbDLEa8K2UHcK}
    \label{QRWireframe}
\end{figure}

\begin{figure}
    \centering
    \includegraphics{exemplos/QRCode/QR youtube.PNG}
    \caption{QR Code - Youtube}
    \underline{https://www.youtube.com/channel/UCi0IphrbwIS3ToNnYDU0mJQ/featured}
    \label{QRYoutube}
\end{figure}

    