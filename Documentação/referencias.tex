\chapter{Referências}
ABINPET. Mercado Pet Brasil 2021. Disponível em: <http://www.abinpet.org.br/download/abinpet_folder_2021.pdf> Acesso em 14 de Abril de 2022.
INSTITUTO PET BRASIL.Censo Pet: 139,3 milhões de animais de estimação no Brasil.2019.Disponível em: <http://institutopetbrasil.com/imprensa/censo-pet-1393-milhoes-de-animais-de-estimacao-no-brasil/> Acesso em: 14 de Abril de 2022.
GOVERNO FEDERAL DO BRASIL. O Brasil registrou mais de 234 milhões de acessos móveis em 2020. Agência Nacional de Telecomunicações. 2021. Disponível em: <https://www.gov.br/pt-br/noticias/transito-e-transportes/2021/05/brasil-registrou-mais-de-234-milhoes-de-acessos-moveis-em-2020> Acesso em: 14 de Abril de 2022.
HEROKU. Heroku Pricing. 2020. Disponível em: <https://www.heroku.com/pricing#app-types-header>.Acesso em: 14 abr. 2022.
RIBEIRO, A. F. de A. Cães Domesticados e os benefícios da interação. Revista Brasileira de Direito Animal, Salvador, v. 6, n. 8, 2014. DOI: 10.9771/rbda.v6i8.11062. Disponível em: https://periodicos.ufba.br/index.php/RBDA/article/view/11062. Acesso em: 14 abr. 2022.
SILVA, N. A.; MARISCO, G. A RELAÇÃO DE ANIMAIS DOMÉSTICOS NA EDUCAÇÃO E SAÚDE. Interfaces Científicas - Saúde e Ambiente, [S. l.], v. 7, n. 1, p. 71–78, 2018. DOI: 10.17564/2316-3798.2018v7n1p71-78. Disponível em: https://periodicos.set.edu.br/saude/article/view/5491. Acesso em: 14 abr. 2022.
SOUZA, A. S. de. Direitos dos animais domésticos: análise comparativa dos estatutos de proteção. Revista de Direito Econômico e Socioambiental. v. 5, n. 1, p. 110–132, 2014. Disponível em: <https://periodicos.pucpr.br/direitoeconomico/article/view/6242>. Acesso em: 14 abr. 2022.
GOOGLE PLAY CONSOLE. Disponível em: <https://play.google.com/console/u/0/signup>.Acesso em: 16 abr. 2022.
APPLE DEVELOPER PROGRAM.Isenção da taxa de assinatura no Apple Developer Program
. Disponível em: <https://play.google.com/console/u/0/signup>.Acesso em: 16 abr. 2022.
Docker. Docker Pricing. 2022. Disponível em: <https://www.docker.com/pricing/>.Acesso em: 16 abr. 2022.
GREGOL, R. E. W. Recursos de escalabilidade e alta disponibilidade para aplicações web.Repositório de Outras Coleções Abertas, p. 17–18, 2011.
ENDEAVOR. Quão longe sua ideia pode ir? Descubra avaliando a escalabilidade dela. 2015. Disponível em: <https://endeavor.org.br/estrategia-e-gestao/escalabilidade/>.Acesso em: 17 abr. 2022.
GOOGLE DEVELOPERS. Disponível em: <https://developer.android.com/docs/quality-guidelines/core-app-quality?hl=pt-br#sc>.Acesso em: 18 abr. 2022.
BRASIL. Lei Geral de Proteção de Dados (2018). 2018. Disponível em: <http://web.archive.org/web/20200915225322/http://www.planalto.gov.br/ccivil_03/_ato2015-2018/2018/lei/L13709.htm>. Acesso em: 18 abr. 2022.
 MARTIN FOWLER.Continuous Integration. 2001.Disponível em: <https://martinfowler.com/articles/continuousIntegration.html>.Acesso em: 18 abr. 2022.
PESSANHA, L.; PORTILHO, F. Comportamentos e padrões de consumo familiar em torno dos “pets”. IV ENEC - Encontro Nacional de Estudos do Consumo, 2008.
GRAF, C. T. O comportamento do consumidor no mercado pet e a relação entre os cães e as pessoas. Universidade Regional do Noroeste do Estado do Rio Grande do Sul, 2016.
CHEN, A.; HUNG, K.; PENG, N. A cluster analysis examination of pet owners ’consumption values and behavior –segmenting owners strategically. Journal of Targeting, Measurement and Analysis for Marketing, v. 20, n. 2, p. 117–132, 2012.
Silvaet.al.Petshow.2021.Disponível em:<https://svn.spo.ifsp.edu.br/svn/a6pgp/S202001-PI/HYVE/Documentos/5-EntregaFina>.Acesso em: 01 mai. 2022.
PRODEST.O uso de aplicativos na sociedade.Disponível em:
<https://prodest.es.gov.br/o-uso-de-aplicativos-na-sociedade#:~:text=Os\%20aplicativos\%20fazem\%20cada\%20vez,op\%C3\%A7\%C3\%B5es\%20de\%20lazer\%20com\%20facilidade.> Acesso em: 03 mai. 2022.
CARVALHO, J. O. F. D. O papel da interação humano-computador na inclusão digital. Transformação, V. 15, n. spe, p. 75-89. Campinas, Dezembro, 2003 .Disponível em: <http://www.scielo.br/scielo.php?script=sci_arttext&pid=S0103-37862003000500004&lng=en&nrm=iso>. Acesso em: 03 mai. 2022.
GLOBOESPORTE.COM. Uso de aplicativos de saúde deve aumentar nos
próximos anos, segundo pesquisa. Grupo Globo. Rio de Janeiro.2019. Disponível
em:<https://globoesporte.globo.com/eu-atleta/noticia/uso-de-aplicativos-de-saude-deve-aumentar-nos-proximos-anos-segundo-pesquisa.ghtml> .  Acesso em: 03 mai. 2022
BRITO, S. Quem usa mais o smartphone: Brasil ou Estados Unidos? Revista Veja digital. Editora Abril. Janeiro de 2021. Disponível em: 
<https://veja.abril.com.br/tecnologia/quem-usa-mais-o-smartphone-brasil-ou-estados-unidos/> . Acesso em: 03 mai. 2022.
SANTOS, A. Brasil, segundo país onde o mercado de aplicativos mais cresce.Terra.2020. Disponível em:<https://www.terra.com.br/noticias/dino/brasil-segundo-pais-onde-o-mercado-de-aplicativos-mais-cresce,1fd9d38aa995ad8ca1243f6c58080f79u2ee8tfj.html#:~:text=Os\%\%20realizados\%20pela\%20empresa,m\%C3\%A9dio\%20real\%20de\%2030\%20apps.&text=Houve\%20um\%20aumento\%20de\%2030,aplicativos\%20do\%20Google\%2C\%20em\%202020> .  Acesso em: 03 mai. 2022.
INSTITUTO PET BRASIL. Censo Pet: 139,3 milhões de animais de estimação no Brasil. 2019. Disponível em: <http://institutopetbrasil.com/imprensa/censo-pet-1393-milhoes-de-animais-de-estimacao-no-brasil/>. Acesso em: 19 mai. 2022.
PUGA, E. F. e. M. G. S. Banco de dados: Implementação em SQL, PL/SQL e Oracle 11g. 1. ed. São Paulo: Pearson Universidades, 2013. Acesso em: 21 mai. 2022.
IBM. Arquitetura de três camadas. 2020. Disponível em: 
<https://www.ibm.com/br-pt/cloud/learn/three-tier-architecture#toc-benefcios--jxGrdA7u>. Acesso em: 22 mai. 2022.
https://www.w3schools.com/js/js_conventions.asp
