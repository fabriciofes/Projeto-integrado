% ---
% Conclusão (outro exemplo de capítulo sem numeração e presente no sumário)
% Dependendo do trabalho desenvolvido ele pode ter uma Conclusão ou Considerações finais
% Para trabalhos de disciplina utilizar Considerações Finais
% ---
\chapter{Considerações Finais}

Esse capítulo tem como intuito descrever o desenvolvimento da aplicação feita pela equipe, demonstrando as dificuldades no decorrer do projeto. Como também informando as lições aprendidas com essa experiência.

\section{Principais Dificuldades}
Desde o início do projeto, foi compreendido pelos componentes que não seria um trabalho fácil, seria necessário auto-disciplina, dedicação  e principalmente, o apoio coletivo da equipe. De maneira inicial, foi difícil, visto que a equipe encontrava-se em níveis diferentes de conhecimento e desenvolvimento. \\
 No início, a dificuldade encontrada, ocorreu quando os integrantes da equipe tinham tempo limitado para o desenvolvimento do projeto. Os integrantes trabalham em tempo integral, fazendo com que, o único tempo disponível seria de noite ou finais de semana. Mesmo reorganizando as funções, cada integrante não teria muito tempo para realizar as funções destinadas a cada um. \\
Imprevistos aconteceram durante a semana, onde interferiu diretamente no projeto. Desse jeito, dois integrantes do grupo saíram, o Kelvin e o Lucas, aumentando a carga de trabalho e a responsabilidade de todos.\\
A segunda dificuldade ocorreu quando os integrantes informaram que teriam conhecimentos básicos com as tecnologias utilizadas, como React Native, PostgreSQL, Expo, etc. Nem todos possuíam uma grande curva de aprendizado e, devido a pouca disponibilidade, isso acabou atrapalhando o desenvolvimento do projeto. Como tentativa de amenizar esse problema, realizamos algumas videochamadas para nos ajudarmos.\\


\section{Lições Aprendidas}
Ao decorrer do projeto, a equipe observou que teria que se comunicar melhor sobre o projeto, para evitar retrabalhos, pois com a falta dessa comunicação fez com que alguns tópicos fossem refeitos como, os Casos de Uso, MER, DER, etc.\\
Os Wireframes passaram a ser refeitos no decorrer do projeto, por conta das correções de funcionalidades do usuário. \\
Após identificar os pontos de melhoria, as mudanças nos padrões de comportamento trouxeram muito mais união à equipe. Observaram-se muitos pontos positivos durante o desenvolvimento, relacionados à empatia, paciência  e esforço de cada integrante durante o trabalho. Desta forma, a equipe teve uma melhora muito grande ao final do projeto. \\


\section{Funcionalidades Futuras}
Por se tratar de um aplicativo de transporte, o projeto tem potencial para ter grandes mudanças no decorrer do tempo e das necessidades que vão aparecer, de melhoria e atualizações que podem abranger leis e normas públicas, como novas funcionalidades para usuários, de modo que o App se torne mais atrativo para os usuário e futuras parcerias. Por exemplo, a inclusão de transporte de Pets usando aviões em viagens nacionais como internacionais.\\
Para o futuro, buscamos ter grande aproveitamento no crescimento do aplicativo para uso não só em território nacional como internacional, gerando assim questões como mais idiomas para o aplicativo, usar servidores externos ou até mesmo parcerias fora do Brasil.\\
Visando o cliente final, é desejado ter uma interface mais simples visando pessoas idosas ou que não possuem grande conhecimento em tecnologia usando aplicativos.\\

