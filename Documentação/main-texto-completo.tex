%% Adaptado a partir de :
%%    abtex2-modelo-trabalho-academico.tex, v-1.9.2 laurocesar
%% para ser um modelo para os trabalhos no IFSP-SPO

\documentclass[
    % -- opções da classe memoir --
    12pt,               % tamanho da fonte
    openright,          % capítulos começam em pág ímpar (insere página vazia caso preciso)
    %twoside,            % para impressão em verso e anverso. Oposto a oneside
    oneside,
    a4paper,            % tamanho do papel. 
    % -- opções da classe abntex2 --schwinn
    % Opções que não devem ser utilizadas na versão final do documento
    %draft,              % para compilar mais rápido, remover na versão final
    paginasA3,  % indica que vai utilizar paginas em A3 
    BIBLATEX,           % indica para utilizar BIBLATEX em vez do abntex2cite
    REFINDENT,          % não fica exatamente no formato da ABNT, mas melhora muito a formatação
                        % não utilizar REFINDENT na versão final
    MODELO,             % indica que é um documento modelo então precisa dos geradores de texto
    TODO,               % indica que deve apresentar lista de pendencias 
    % -- opções do pacote babel --
    english,            % idioma adicional para hifenização
    brazil              % o último idioma é o principal do documento
    amsmath,
    latexsym,
    graphicx,
    babel,
    epigraph,
    mathtools,
    amssymb,
    afterpage,
    calc,
    setspace,
    ]{ifsp-spo-inf-cemi} % ajustar de acordo com o modelo desejado para o curso

\titulo{Carrara Pets}

\renewcommand{\imprimirautor}{
\begin{tabular}{lr}
Davi Henrique Silva de Oliveira & SP3013081 \\
Fabricio Ernesto dos Santos & SP3013171 \\
Guilherme Santana Reis de Souza & SP3013278 \\
Jose Ronaldo da Silva Junior & SP3013197 \\
Lorena Moreira Bezerra & SP3013316 \\
\end{tabular}
}


\disciplina{PI1A5 - Projeto Integrado I}

\preambulo{Modelo canônico de trabalho monográfico acadêmico em conformidade com
as normas ABNT apresentado à comunidade de usuários \LaTeX.}

\data{29/03/2022}

% Definir o que for necessário e comentar o que não for necessário
% Utilizar o Nome Completo, abntex tem orientador e coorientador
% então vão ser utilizados na definição de professor
\renewcommand{\orientadorname}{Professor:}
\orientador{José Braz de Araujo}
\renewcommand{\coorientadorname}{Professor:}
\coorientador{Marcelo Tavares de Santana}


% ---


% informações do PDF
\makeatletter
\hypersetup{
        %pagebackref=true,
        pdftitle={\@title}, 
        pdfauthor={\@author},
        pdfsubject={\imprimirpreambulo},
        pdfcreator={LaTeX with abnTeX2 using IFSP model},
        pdfkeywords={abnt}{latex}{abntex}{abntex2}{IFSP}{\ifspprefixo}{trabalho acadêmico}, 
        colorlinks=true,            % false: boxed links; true: colored links
        linkcolor=blue,             % color of internal links
        citecolor=blue,             % color of links to bibliography
        filecolor=magenta,              % color of file links
        urlcolor=blue,
        bookmarksdepth=4
}
\makeatother
% --- 

% carregando aqui referencias quando utilizando BIBLATEX
\IfPackageLoaded{biblatex}{%
\addbibresource{referencias.bib}
\addbibresource{exemplos/abntex2-doc-abnt-6023.bib}
}{}

% ----
% Início do documento
% ----
\begin{document}



% Retira espaço extra obsoleto entre as frases.
\frenchspacing 

%somente para o exemplo, fica primeiro
%\todo[inline]{Remover texto informativo inicial}
%\input{00-info}

% -- lista de pendencias gerada pelo todonotes
% -- altere opções do usepackage para remover na versão final....
%\listoftodos
%\todo[inline]{remover lista de todo da versão final...}
\newpage


% ----------------------------------------------------------
% ELEMENTOS PRÉ-TEXTUAIS
% ----------------------------------------------------------
%\pretextual

% ---
% Capa
% ---
\imprimircapa

%\newcounter{todocounter}
%\newcommand{\todonum}[2][]
%{\stepcounter{todocounter}\todo[#1]{\thetodocounter: #2}}


%\todonum[inline]{ajustar titulo do trabalho}
%\todonum[inline]{ajustar autor}
%\todonum[inline]{ajustar data}
%\todonum[inline]{ajustar preambulo}
%\todonum[inline]{ajustar curso}
%\todonum[inline]{ajustar disciplina}
%\todonum[inline]{ajustar departamento}
%\todonum[inline]{ajustar orientador/coorientador/professor(es)}
% ---

% ---
% Folha de rosto
% (o * indica que haverá a ficha bibliográfica)
% ---
\imprimirfolhaderosto
%\imprimirfolhaderosto*
% ---

% Quando registrado na biblioteca
%\input{pre-fichacatalografica}

%Caso necessário
%\input{pre-errata}

%Obrigatório para trabalhos com bancas oficiais
%\input{pre-aprovacao}

% ---- opcionais 
\input{pre-dedicatoria}
% ---
% Agradecimentos
% ---
\begin{agradecimentos}
Agradecemos todas as pessoas que nos apoiaram para a construção deste trabalho. Após inúmeras dificuldades para a realização deste documento, agradecemos à equipe e pelos entes de cada um que nos ajudou a enfrentar as dificuldades e nos incentivar a ir adiante desta documentação. 
Agradecemos também aos professores José Braz e Marcelo Tavares, os orientadores, o qual nos orientou e nos auxiliou para conseguir concluir este trabalho.
Agradecemos também aos nossos animais que estão sempre presentes em nossas vidas, nos alegrando e fazendo companhia em dias difíceis. 

\end{agradecimentos}
% ---
% ---
% Epígrafe
% ---
\begin{epigrafe}
    \vspace*{\fill}
    \begin{flushright}
        \textit{``Quem tem um quatro patas na vida, tem tudo.''\\
        (Denise Campos)}
    \end{flushright}
\end{epigrafe}
% ---

% -- resumo obrigatório
% ---
% RESUMOS
% ---

% resumo em português
\setlength{\absparsep}{18pt} % ajusta o espaçamento dos parágrafos do resumo
\begin{resumo}
    
    Segundo dados da ABINPET (2021)(LINK AQUI), o Brasil ocupa a sétima posição no Ranking(LINK AQUI) mundial de gastos com animais de estimação. Além disso, no ano de 2019 existiam cerca de 144 milhões de animais domésticos, em sua maioria cães e gatos, e cerca de vinte e cinco por cento do faturamento do ramo Pet(LINK AQUI) é com serviços de Pet Care e Pet Vet.  Tendo em vista que uma das necessidades dos tutores de Pets(LINK AQUI) é o transporte desses seres para consultas, exames, creches caninas, e passeios, etc.O objetivo deste projeto é criar um aplicativo gratuito, de transporte de animais, disponível para celulares Android e IOS pois no mercado nacional não encontramos muitas opções de transporte de animais. Para a criação da aplicação foi realizada uma análise de concorrência, definição de público alvo, verificação de projetos similares e norteadores na literatura, delimitação de tecnologias e linguagens de programação para o desenvolvimento,  estudo de métodos para viabilidade financeira e   captação de receitas. Além disso, no escopo do projeto foram definidas as provas de conceito, produto  mínimo viável, arquitetura da aplicação, banco de dados,   diagrama de classes, diagrama de sequência, critério de segurança, Privacidade e legislação, critérios de log, code convertions, processo de integração contínua, e manutenibilidade da aplicação. Após análises, os autores concluíram que o projeto é totalmente viável e rentável. Para o futuro, espera-se que haja melhorias conforme a necessidade, de modo que o App se torne mais atrativo para os usuários e futuras parcerias.

 \textbf{Palavras-chaves}: latex. abntex. editoração de texto.
\end{resumo}

% resumo em inglês
\begin{resumo}[Abstract]
 \begin{otherlanguage*}{english}

   This is the english abstract.

   \vspace{\onelineskip}

   \noindent 
   \textbf{Keywords}: latex. abntex. text editoration.
 \end{otherlanguage*}
\end{resumo}


% ---
% inserir lista de ilustrações
% ---
\pdfbookmark[0]{\listfigurename}{lof}
\listoffigures*
\cleardoublepage
% ---

% ---
% inserir lista de tabelas
% ---
\pdfbookmark[0]{\listtablename}{lot}
\listoftables*
\cleardoublepage
% ---

% ---
% inserir lista de quadros
% ---
\pdfbookmark[0]{\listofquadrosname}{loq}
\listofquadros*
\cleardoublepage
% ---

\input{pre-siglas}

% \input{pre-simbolos}


% ---
% inserir o sumario
% ---
\pdfbookmark[0]{\contentsname}{toc}
\tableofcontents*
\cleardoublepage
% ---


% ----------------------------------------------------------
% ELEMENTOS TEXTUAIS
% ----------------------------------------------------------
\textual


% ----------------------------------------------------------
% Introdução
% ----------------------------------------------------------
\chapter[Introdução]{Introdução}
Ao observar alguns aspectos da sociedade moderna, os animais domésticos se tornaram figuras essenciais em diversos lares, ocupando o lugar de amigo, filho ou companheiro. Segundo (Ribeiro, 2011)(LINK AQUI) animais domésticos podem trazer sentimentos de lealdade, companheirismo e confiabilidade a humanos que nem sempre são construídos com relações entre indivíduos da mesma espécie. De acordo com (Silva e Marisco, 2018)(LINK AQUI) esses indivíduos quando inseridos na realidade humana, podem trazer diversos efeitos terapêuticos, psicossociais e fisiológicos. \\
 Segundo dados da ABINPET (2021)(LINK AQUI), o Brasil ocupa a sétima posição no Ranking(LINK AQUI) mundial de gastos com animais de estimação. Além disso, no ano de 2019 existiam cerca de 144 milhões de animais domésticos, em sua maioria cães e gatos, e cerca de vinte e cinco por cento do faturamento do ramo Pet é com serviços de Pet(LINK AQUI) Care e Pet Vet. \\
            Com o mundo globalizado, a utilização de Smartphones(LINK AQUI) com aplicativos para tarefas simples vem sendo amplamente difundida pela sociedade. \\
Em uma matéria do Site(LINK AQUI) do governo federal brasileiro (2021) diz: \underline{\textit{“(...)após a popularização desses aplicativos, os consumidores passaram a abrir mão de acessos em diversas prestadoras, tornando-se clientes de uma única empresa(...)”.}} Portanto faz-se necessário que as soluções para tutores de cães e gatos tenham opções seguras e que supram suas necessidades para o cuidado do seu animal. \\
 Tendo em vista que uma das necessidades dos tutores de Pets(LINK AQUI) é o transporte desses seres para consultas, exames, creches caninas, e passeios, este trabalho apresenta, documenta e contextualiza as etapas de desenvolvimento do aplicativo Carrara Pets.\\



\newpage
\section{Objetivos}

\subsection{Objetivo Geral}
O objetivo deste projeto é criar um aplicativo gratuito, de transporte de animais, disponível para celulares  com sistemas Android e IOS,pois são os mais utilizados atualmente.\\

\subsection{Objetivo Específico}
O nosso objetivo específico é que a nossa aplicação funcione de modo similar aos aplicativos de transporte particular humano (UBER, 99, etc.), pois os motoristas com habilitação poderão se credenciar no Carrara Pets(LINK AQUI) e deslocar Pets da subespécie Canis lupus familiaris(LINK AQUI) e da espécie Felis catus(LINK AQUI) com conforto e confiança para os destinos escolhidos pelos clientes.\\
Ter a segurança como uma das nossas prioridades: cães serão transportados utilizando cintos/ cadeiras de segurança e gatos serão transportados em caixas próprias para transporte. Além disso, os carros serão higienizados a cada corrida, os tutores poderão escolher a opção do animal de estimação viajar sozinho ou acompanhado e os motoristas serão avaliados pelos clientes na plataforma.\\


\section{Justificativa}
Atualmente no mercado nacional não encontramos muitas opções de transporte de animais, e os existentes não suprem todas as necessidades de seus usuários, criando assim uma lacuna na melhoria contínua já que não houve muita evolução desses aplicativos.\\
Segundo os dados de 2021 do Instituto Pet Brasil, informa que há cerca de 143,53 milhões de animais de estimação, sendo 58,1 milhões de cachorros, 27,1 milhões de gatos e 58,33 milhões entre outros animais de estimação (INSTITUTO PET BRASIL, 2019 - Gráfico 1).(LINK AQUI)
\begin{figure}
    \centering
    \includegraphics{exemplos/diagramas/População_Animais.jpeg}
    \caption{A população de animais no Brasil no ano de 2021.}
    \label{fig:População_Animais}
    \fonte{Instituto Pet Brasil apud IBGE}
\end{figure}
\\
Considerando o grande número de pessoas com Pets(LINK AQUI) e o carinho que possuem por eles, criou-se   um certo medo de como seus animais serão tratados. Segundo o (Graf , 2016)(LINK AQUI) os donos de animais de estimação, no momento de escolher profissionais para cuidar de seus Pets(LINK AQUI), possuem muitos receios, no que tange a qualidade do serviço e o cuidado do profissional, além disso, o preço do serviço (Gráfico 2)(LINK AQUI).
Um método muito efetivo de conhecer a qualidade de um serviço é por indicação que uma pessoa recebe de um amigo ou conhecido,  nesse âmbito, a aplicação busca disponibilizar avaliações dos parceiros feitas por outros usuários. 
\begin{figure}
    \centering
    \includegraphics{exemplos/diagramas/graf_analise.jpeg}
    \caption{Principais motivos de escolha de Pet Shops. Os números indicam o número das perguntas, no questionário original da referência Graf (2016)(LINK AQUI)}
    \label{fig:graf_analise}
    \fonte{Graf (2016)(LINK AQUI)}
\end{figure}
\\

\subsection{Análise de concorrência}
Uma das ferramentas importantes no design de projetos é analisar empreendimentos similares e verificar os pontos positivos e negativos para justificar a criação de uma nova proposta. Entre os concorrentes atuais temos:
\begin{itemize}
    \item \textbf{PetDrive:}
        \begin{enumerate}
            \item \textbf{Pontos positivos:}
            Um dos aplicativos mais conhecidos pelo transporte de animais de estimação com seus donos, tendo como um de seus diferenciais o agendamento e equipamentos de transporte de animais dentro do carro disponibilizado pela empresa.
            \item \textbf{Pontos negativos:}
            Não disponibiliza a opção do cliente usar o seu próprio equipamento e não há a opção de viagens de animais desacompanhados.
    \end{enumerate}
    \item \textbf{Uber Pets}:
    \begin{enumerate}
            \item \textbf{Pontos positivos:}
            Conhecido dentro e fora do Brasil, interface intuitiva.
            \item \textbf{Pontos negativos:}
            Os carros não são adaptados para transporte de animais, não transportam o pet sozinho.
    \end{enumerate}
    \item \textbf{Mr. taxidog:}
    \begin{enumerate}
            \item \textbf{Pontos positivos:}
            Empresa antiga  no mercado, com profissionais qualificados e preparados no transporte de cachorros.
            \item \textbf{Pontos negativos:}
            Não é possível transportar outros pets como gatos e pássaros. Funciona somente no estado do Paraná.
    \end{enumerate}
    \item \textbf{Driver dog:}
    \begin{enumerate}
            \item \textbf{Pontos positivos:}
            Empresa com mais de 14 anos no mercado. Possui profissionais qualificados para lidar com cães, e transporte por longas distâncias.
            \item \textbf{Pontos negativos:}
            Não é possível transportar outros seres como gatos, coelhos,etc.
    \end{enumerate}
    \item \textbf{Dog hero:}
    \begin{enumerate}
            \item \textbf{Pontos positivos:}
            Possui profissionais qualificados na área de saúde animal, hospedagem de pets e creche. Está disponível em todo o Brasil.
            \item \textbf{Pontos negativos:}
            Aplicativo pouco intuitivo de transporte com veículo.
    \end{enumerate}
\end{itemize}

Analisando todos os concorrentes vemos que existe espaço para competitividade e crescimento de um novo produto. Portanto, o Carrara Pets busca trazer uma nova experiência para seus clientes , trazendo inovação, facilidade e mais funcionalidades para o cliente e seu pet, não se propondo a só transportar e cuidar dos pets, mas também buscar parcerias que agreguem mais valor para o negócio e satisfação ao consumidor final.

\begin{quadro}[thb]
\centering
\ABNTEXfontereduzida
\caption{Comparativo entre concorrentes}
\label{quadro-poluido-limpo-desalinhado}
\begin{tabular}{|l|l|l|l|l|l|l|}
\hline
\thead{Recursos} & \thead{Pet\\Driver} & \thead{Uber\\Pets} & \thead{Carrara\\Pets} & \thead{Mr.\\Taxidog} & \thead{Driver\\Dog} & \thead{Dog\\Hero}  \\
\hline
Transportar animais & \circlemark & \circlemark & \circlemark & \circlemark & \circlemark & \\
\hline
Corridas agendadas &\circlemark& &\circlemark& & &\circlemark\\
\hline
Serviços prestados pelo motorista & & &\circlemark& & &\\
\hline
Acompanhamento de viagem  &\circlemark&\circlemark&\circlemark& & &\circlemark\\
\hline
Hospedagem & & &\circlemark& & &\circlemark\\
\hline
Creche & & &\circlemark& & &\circlemark\\
\hline
Viagem compartilhada & & & &\circlemark& & \\
\hline
\end{tabular}
\fonte{Os autores.}
\end{quadro}

\section{Estrutura do Documento}
O presente documento está estruturado em 5 capítulos. O Capítulo 1(LINK AQUI), traz a contextualização do tema do projeto e a solução proposta, além de mostrar o objetivo geral e o objetivo específico, que deseja-se conquistar no desenvolvimento da aplicação. Ademais, possui a justificativa para a solução e uma análise de concorrência, demonstrando quais seriam os recursos disponíveis na plataforma Carrara Pets. Para finalizar, o capítulo contém esta seção para explicar a estrutura do documento. \\
No Capítulo 2(LINK AQUI), Revisão da Literatura, reúne-se as referências que forneceram embasamento para o trabalho. Neste capítulo, o leitor contará com temas referentes à aplicativo e público alvo, pontos principais da aplicação.\\
O Capítulo 3(LINK AQUI), Gerenciamento do Projeto, possui seções direcionadas a organização da equipe, falando sobre a metodologia que está sendo utilizada, links que contém algumas informações, como o canal do YouTube, blog da equipe e entre outros. \\
O Capítulo 4(LINK AQUI), Desenvolvimento do Projeto, aborda sobre as tecnologias utilizadas, viabilidade da aplicação, mostra os itens do escopo do projeto, apresenta a arquitetura da aplicação, a modelagem de dados e o diagrama de classes; explica sobre os padrões de projeto, sobre a segurança da informação e outros tópicos referentes ao desenvolvimento da aplicação.\\
Por fim, o Capítulo 5(LINK AQUI) é composto pelo resumo do projeto, explicitando as dificuldades de implementar o Carrara Pets e os obstáculos que apareceram durante o desenvolvimento e as funcionalidades futuras que poderão ser adicionadas à aplicação.\\
Além destes capítulos, o documento contém uma seção de Apêndice, com os documentos extras criados pela equipe ao decorrer do desenvolvimento, que acrescentam informações e melhoram o entendimento do projeto.


\newpage
\section{Processos}
\begin{itemize}
    \item Acompanhamento do veículo 
    \item Diagnóstico do atendimento (Veterinário)
    \item Possível cobrança em casos de compra de remédio
    \item Definição de rota para viagem
    \item Método de fidelidade (Pontos)
    \item Salvar acontecimentos da corrida em forma do tempo
    \item Fidelidade de motorista “X” pet (rotina)
    \item Compatibilidade do porte do pet com dimensões do veículo
    \item Validação do cadastro do motorista (Nome, Idade, Carro, Placa, CNH, CPF, Pré-entrevista com psicólogo(a))
    \item Cadastro de características
    \item Cadastro do pet (Tamanho, Peso, Porte, Temperamento)
    \item Cadastro do dono (Nome, CPF, Idade, Telefone, Endereço...)

\end{itemize}



% ---
% Capitulo de revisão de literatura
% ---
\chapter{Revisão da Literatura}

Na antiguidade, os seres humanos e outros animais caçavam devido a seu instinto, porém em algum momento, os seres humanos entenderão que ao domesticar esses animais poderiam ter maior efetividade no seu dia a dia. No decorrer do tempo, o ser humano descobriu que ao coexistir com animais, eles os tornavam mais fortes e felizes, foram inseridos cada vez mais em seu cotidiano até se tornarem parte da família, como seres de estimação, companheiro fiel e um apoio emocional (PESSANHA; PORTILHO, 2008)(LINK AQUI).
Olhando a sociedade moderna, onde esses animais se tornaram extremamente valorizados, de modo que chegam a substituir filhos. Muitas pessoas desistem de ter filhos para ter um animal tratado como tal (PESSANHA; PORTILHO, 2008)(LINK AQUI).

\section{Aplicativos}
Com o desenvolvimento da tecnologia o uso de aparelhos celulares e seus aplicativos têm se tornado essenciais no cotidiano das pessoas, por inúmeros motivos como facilidade para acessar contas bancárias, solicitar transporte, pedir uma refeição etc. Tudo isso de maneira rápida e extremamente acessível. Segundo pesquisas, o Brasil possui um grande potencial no âmbito de dispositivos móveis, cerca de 29\% da população utiliza aplicativos para atividades corriqueiras. Outra pesquisa mostra que as pessoas do Brasil ficam em torno de 3 horas em frente ao celular diariamente, superando países como EUA. (CARVALHO, 2003; GLOBOESPORTE.COM, 2019; SANTOS, 2020; BRITO, 2021)(LINK AQUI).\\
No projeto, a ideia de criar um aplicativo ao invés de um WebApp ou uma aplicação Web, deu-se o intuito de ter maior credibilidade e funcionalidade na aplicação, além de usar uma baixa quantidade de dados móveis para o cliente final.

\section{Público alvo}
Silva (2021) apud Graf (2016) Diz que os animais são atualmente tratados como membros da família. Segundo Silva et.al. (2021), a relação entre seres domesticados e seus donos pode ser separada em:
\begin{itemize}
    \item \underline{" Afeto: O dono usa mais serviços/produtos de alta qualidade, como enfeites, xampu\\, roupas etc.;" }
    \item \underline{"Interação: Onde o dono contrata serviços de adestramento para adequar o animal \\ao seu estilo de vida e produtos para o bem-estar dele;"}
    \item \underline{"Substituição humana: O consumidor substitui as relações humanas, como filhos ou \\amigos, pelo animal de estimação, pagando por atividades de tratamentos \\veterinários de luxo, adestramento e atividades geralmente associadas a relações humanas,\\ como funerais.”}
\end{itemize}

% ---

% ---

% ---

\chapter{Gerenciamento do Projeto}
Neste capítulo é apresentado o gerenciamento do projeto e todas as etapas seguidas, dentre estas destacam-se as reuniões e Sprint.
\\

\section{Metodologia}
A metodologia de gerenciamento do projeto utilizada é o Scrum, um Framework que tem como objetivo viabilizar o gerenciamento ágil de projetos de software, através de práticas e técnicas que garantem a comunicação efetiva dos integrantes, facilitando o aprendizado e a melhoria contínua das pessoas envolvidas no projeto. \\
Seguindo esse \textit{Framework(LINK AQUI)}, o projeto possui um conjunto de atividades que devem ser executadas em iterações semanais, as quais são chamadas de \textit{Sprint(LINK AQUI)}. De início, são criadas atividades e funcionalidades para o projeto, que são chamadas de \textit{Product Backlog}. Ao iniciar uma \textit{Sprint(LINK AQUI)}, é realizado uma reunião com a equipe para definir quais funcionalidades devem ser implementadas do \textit{Product Backlog}, essa reunião é conhecida como\textit{ Sprint Planning}, as tarefas selecionadas pela equipe são chamadas de \textit{Sprint Backlog} e as funcionalidades escolhidas são definidas como \textit{Users Stories}. Ao final da Sprint(LINK AQUI), a equipe apresenta as funcionalidades implementadas em outra reunião chamada \textit{Sprint Review}, e após isso, o ciclo se reinicia.
\\

\section{Organização da equipe}
A equipe é composta por 5 membros que foram definidos no primeiro dia de aula, onde houve a apresentação da disciplina. Com a equipe formada foi decidido o papel de cada pessoa dentro do projeto de acordo com a experiência de cada um. \\
O Quadro 2 apresenta o papel de cada pessoa no projeto. Vale ressaltar que os integrantes não farão atividades apenas voltadas ao que está no quadro, mas que darão um foco maior nas atividades descritas.
\begin{quadro}[thb]
\centering
\ABNTEXfontereduzida
\caption{Organização da equipe}
\label{quadro-poluido-limpo-desalinhado}
\begin{tabular}{|l|l|l|l|l|l|}
\hline
\thead{Atividades} & \thead{Davi} & \thead{Fabricio} & \thead{José} & \thead{Lorena} & \thead{Guilherme}\\
\hline
Arquitetura& & & \circlemark & & \circlemark \\
\hline
Banco de dados& & \circlemark & \circlemark & & \circlemark \\
\hline
Blog& \circlemark & & & \circlemark & \\
\hline
Código Latex& & \circlemark & & & \\
\hline
Desenvolvimento back-end& & \circlemark & \circlemark & & \circlemark \\
\hline
Desenvolvimento front-end& \circlemark & & & \circlemark & \\
\hline
Documentação& \circlemark & \circlemark & \circlemark & \circlemark & \circlemark \\
\hline
Gerência& & & & \circlemark & \\
\hline
Youtube& \circlemark & & & & \\
\hline
\end{tabular}
\caption{Organização da equipe}
\fonte{Os autores.}
\end{quadro}

O Davi é responsável por realizar as atualizações do blog(LINK AQUI) semanalmente e das apresentações no Youtube e também fará parte do desenvolvimento front-end do projeto.\\ 
O Fabrício é responsável pelo banco de dados e diagramas e é o líder na área de LATEX o trabalho dele será direcionar, ensinar e revisar os códigos LATEX da documentação. \\
O José está encarregado de cuidar do banco de dados juntamente com Fabrício além de ser responsável pelo design de arquitetura e preparar o ambiente de desenvolvimento na primeira etapa do projeto. No decorrer do desenvolvimento ele fará a programação back-end juntamente com o Fabrício.  \\
A Lorena é a gerente de projetos e tem como atividade liderar, organizar, estimular a equipe a sempre trabalharem juntos e compartilhar das dificuldades. Atua no desenvolvimento front-end e realiza as atualizações do blog juntamente com o Davi semanalmente. \\
O Guilherme é responsável pelo design da aplicação, paleta de cores, warframe e suporte na parte de back-end(LINK AQUI) do projeto.\\
Todos os integrantes são responsáveis pelo desenvolvimento da documentação e pela realização dos testes unitários.
\\

\section{Comunicação da equipe}
A equipe disponibiliza vários links com os recursos aqui produzidos, nesses links encontram-se informações sobre o trabalho, documentos, relatos semanais e vídeos de desenvolvimento do projeto(Reuniões e apresentações).  \\
	A \ref{QRYoutube} indica o link para o Youtube utilizado pela equipe para compartilhamento de reuniões, apresentações e etc.\\
Os links das \ref{QRTrello} é referente ao Trello, ferramenta de gerenciamento utilizado pela equipe no início do projeto, nesses sites é possível encontrar o planejamento do projeto.\\
A \ref{QRSVN} indica o link para o repositório Subversion utilizado pela equipe para compartilhamento de arquivos e entrega de documentos.\\
Como requisito da disciplina, o projeto deve ser relatado em um blog, disponível em \ref{QRBlog}.\\
Como objeto adicional para o projeto foi criado um wireframe das telas da aplicação contextualizando, disponível em \ref{QRWireframe}.\\

\newpage
\begin{figure}
    \centering
    \includegraphics{exemplos/QRCode/QR blog.PNG}
    \caption{QR Code - Blog}
    \underline{https://carrarapets.blogspot.com/2022/04/reuniao-com-o-professor-na-terca.html?zx=8b6f737ec7e5c039}
    \label{QRBlog}
\end{figure}
    
\begin{figure}
    \centering
    \includegraphics{exemplos/QRCode/QR SVN.PNG}
    \caption{QR Code - SVN}
    \underline{https://svn.spo.ifsp.edu.br/svn/a6pgp/S202201-PI-NOT/Grupo4/}
    \label{QRSVN}
\end{figure}

\begin{figure}
    \centering
    \includegraphics{exemplos/QRCode/QR trello.PNG}
    \caption{QR Code - Trello}
    \underline{https://trello.com/invite/b/VwkOBdgD/8de44bebb8430adfd8af3871dabc7d4c/projeto-pi1}
    \label{QRTrello}
\end{figure}
    
\begin{figure}
    \centering
    \includegraphics{exemplos/QRCode/QR wireframe.PNG}
    \caption{QR Code - Wireframe}
    \underline{https://whimsical.com/wireframe-tcc-Hi9cvRoYkUbDLEa8K2UHcK}
    \label{QRWireframe}
\end{figure}

\begin{figure}
    \centering
    \includegraphics{exemplos/QRCode/QR youtube.PNG}
    \caption{QR Code - Youtube}
    \underline{https://www.youtube.com/channel/UCi0IphrbwIS3ToNnYDU0mJQ/featured}
    \label{QRYoutube}
\end{figure}

    
\chapter{Desenvolvimento do projeto}
Neste capítulo é apresentado itens relacionados ao desenvolvimento, descrevendo as tecnologias utilizadas, escopo do projeto, dividido entre Prova de Conceito (POC)(LINK AQUI), MVP(LINK AQUI) e entrega final, arquitetura da aplicação, diagramas de modelagem e entre outros. \\
\section{Técnologias utilizadas}
As tecnologias que serão utilizadas para o desenvolvimento do aplicativo tem como o principal objetivo é deixar a aplicação flexível e viável com diferentes plataformas.\\
As ferramentas escolhidas são:
\begin{itemize}
    \item \textbf{Vscode :}
    Como IDE será usado o Vscode por sua praticidade no desenvolvimento de aplicativos e programas, além de ser super compatível com Javascript e Typescript.
    \item \textbf{JavaScript/TypeScript :}
    Como linguagem complementar do React native, além de usar o typescript para melhor estruturar o projeto no desenvolvimento.
    \item \textbf{React Native :}
    Será usado para desenvolver o aplicativo, visando atender sistemas Android e IOS de forma nativa, sendo totalmente compatível com Heroku.
    \item \textbf{Expo :}
    Consiste num conjunto de ferramentas voltadas para facilitar o desenvolvimento de aplicações mobile. Fornece diversos simplificadores para desenvolvimento e testes, sendo bem simples e prático no desenvolvimento da aplicação.
    \item \textbf{Express :}
    Como framework será utilizado o express tendo em conta que será usado Node.js na aplicação desenvolvida do projeto, e  por ser um framework minimalista e flexível.
    \item \textbf{PostgreSQL :}
    Por ser um banco de dados bem completo oferecendo a possibilidade de armazenar dados não relacionais, ele também é de fácil hospedagem no Heroku, disponibilizando a plataforma para uso gratuito em desenvolvimento.
    \item \textbf{Heroku :}
    Para a hospedagem é usado a Platform as a Service (PaaS)(LINK AQUI) Heroku, que oferece gratuitamente hospedagem que oferece para contas free 550 horas mensais de atividade de ‘app type’ (métrica de processamento da plataforma).
    \item \textbf{SVN :}
    Como uma das  ferramentas de versionamento, para gerenciar a documentação do projeto, geração de vídeos como os commits e modificações  realizadas pelos integrantes do grupo.
    \item \textbf{Docker :}
    Visando ter uma ferramenta de gerenciamento da infraestrutura aplicada será utilizado o Docker, criando uma maior agilidade na manutenção e modificação da aplicação.
    \item \textbf{Google Maps API :}
    O uso da API(LINK AQUI) do \textit{Google maps}, que disponibiliza um meio de cálculo de rotas, rastreio para acompanhar rotas, para deixar a aplicação mais completa.
\end{itemize}
    O aplicativo precisa fornecer recursos de integração com outras aplicações, isto é, autenticação utilizando API(LINK AQUI) do Google, facilitando o acesso e a criação de cadastros dos usuários.

\section{Viabilidade financeira de mantimento da aplicação}
A aplicação estará alocada no serviço de hospedagem do Heroku(LINK AQUI), que em seu plano gratuito, armazenará aplicativos não comerciais, MVPs e projetos pessoais (HEROKU, 2020)(LINK AQUI). Porém, a hospedagem é limitada, e por isso, considerando que a aplicação terá fins comerciais e futuramente será necessário aumentar a demanda de usuários, de modo que a equipe migraram para o serviço Heroku(LINK AQUI) padrão, que custa por volta de cinquenta dólares (US\$ 50) correspondente a duzentos e trinta e cinco reais e doze centavos (R\$ 235,12) mensais (HEROKU, 2020)(LINK AQUI).\\
Visando o melhor banco de dados foi identificado que o PostgreSQL(LINK AQUI) é o melhor primeiramente por ser gratuito, totalmente compatível com o Heroku(LINK AQUI), porém tendo a possibilidade de fazer doações a melhoria contínua do banco. Após o término do projeto será migrado para o plano standard com o custo de cinquenta dólares (US\$50) correspondente a duzentos e trinta e cinco reais e doze centavos (R\$235,12) mensais (HEROKU, 2020)(LINK AQUI).\\
O app será disponibilizado na Play Store(LINK AQUI) possuindo um custo fixo pago somente uma vez no valor de vinte e cinco dólares (US\$25), correspondente a cento e dezessete e cinquenta e cinco centavos  (R\$117,55 cotação atual) aproximadamente (GOOGLE PLAY CONSOLE,2022)(LINK AQUI).\\
O app também será disponibilizado na Apple Store(LINK AQUI) possuindo um custo fixo pago somente uma vez no valor de noventa e nove dólares (US\$99), correspondente a quatrocentos e sessenta e cinco reais e cinquenta e um centavos (R\$465,51 cotação atual) aproximadamente (APPLE DEVELOPER PROGRAM,2022)(LINK AQUI).\\
Para ter um maior controle da infraestrutura do projeto, agilidade de criação, manutenção e modificações de serviços o Docker(LINK AQUI) é a melhor escolha no plano gratuito, pensando em levar a aplicação adiante após a finalização do projeto, alterar o plano para o pro no valor de cinco dólares (US\$5) correspondente a vinte e tres reais e cinquenta e um centavos (R\$23,51 cotação atual) aproximadamente mensal (DOCKER,2022)(LINK AQUI)\\
Para o gerenciamento de Log, será utilizado o Papertrail(LINK AQUI), que é um software adicional no Heroku, disponibilizado no plano gratuito de dez megabytes por sete dias, porém pensando e no seguimento da aplicação, após o término do projeto, será alterado para o Plano Fixo por oito dólares (US\$8) correspondente a trinta e sete reais e vinte centavos (R\$37,20 cotação atual) aproximadamente mensal.\\
Com relação aos custos, a seguir há dois quadros que informam o custo da aplicação. O primeiro quadro (Quadro 3)(LINK AQUI) , mostra os custos durante o desenvolvimento da aplicação, também foi colocado o valor da licença da Play Store(LINK AQUI) e Apple Store(LINK AQUI) somente para ilustrar, por ser somente para o projeto a aplicação não será disponibilizada no momento.\\
O segundo quadro (Quadro 4)(LINK AQUI), informa o custo da aplicação mensal após o término do projeto, onde o grupo dá seguimento no projeto com fins lucrativos.

\newpage
\begin{figure}
    \centering
    \includegraphics{exemplos/diagramas/Custo com despesas da aplicação no desenvolvimento.jpeg}
    \caption{Custo com despesas da aplicação no desenvolvimento}
    \label{fig:Custo com despesas da aplicação no desenvolvimento}
    \fonte{Os autores.}
\end{figure}\\

\begin{figure}
    \centering
    \includegraphics{exemplos/diagramas/Custo com despesas da aplicação após o término do projeto.jpeg}
    \caption{Custo com despesas da aplicação após o término do projeto}
    \label{fig:Custo com despesas da aplicação após o término do projeto}
    \fonte{Os autores.}
\end{figure}

\subsection{Captação de Receita}
Para a obtenção de fundos para a empresa e para o aplicativo, começaremos buscando parcerias, para incluir na aplicação, por esse meio, podemos fechar uma porcentagem, por indicação que pode variar de 5\% a 15\% do valor, inicialmente será variado, mas ao decorrer do tempo iremos ajustar os valores conforme o crescimento da empresa.\\
Com o tempo, prestaremos um serviço onde o parceiro poderá solicitar uma viagem em nosso aplicativo para buscar ou levar o seu animal em locais específicos, criando assim, uma parceria significativa, pois utilizaremos esse meio para ajudar a empresa terceira crescer, como também divulgaremos o nosso aplicativo para que outras empresas possam nos contratar para realizar esse tipo serviço. \\
Além das parcerias, iremos usar a opção de serviço de passeio do aplicativo para obter receita, na qual cobraremos o valor fixo de cinquenta reais (R\$50,00) a cada passeio solicitado pelo cliente. Também terá um serviço de compras dentro do aplicativo, onde forneceremos a opção do cliente disponibilizar o próprio alimento do animal ao motorista ou terá um campo específico para o cliente adicionar a compra da refeição do animal dentro do aplicativo e no final da corrida, será enviado um comprovante com os valores prestados pelo motorista. \\
E por último terá um método de viagem padrão, na qual teremos uma tarifa base de acordo com cada região, como o valor mínimo a receber, valor por minuto de viagem, valor por quilômetro rodado e o valor de serviço (do aplicativo) de 25\% que será cobrado em cada viagem ou serviço. \\
No Quadro 5(LINK AQUI) está demonstrando a cobrança da viagem padrão do passageiro e do motorista.\\

\newpage
\begin{figure}
    \centering
    \includegraphics{exemplos/diagramas/Cálculo de viagem padrão.jpeg}
    \caption{Cálculo de viagem padrão}
    \label{fig:Cálculo de viagem padrão}
    \fonte{Os autores.}
\end{figure}\\
Para exemplificar a cobrança no quadro acima (Quadro 5)(LINK AQUI) foi realizado a fórmula: 
\mathrm{\textit{ Valor base + (Valor KM * KM) + (Valor minuto * minuto) = > Valor mínimo}}
\begin{itemize}
    \item Valor Base: é o valor fixo para se calcular a viagem;
    \item Valor KM: valor que será usado para calcular a quilometragem;
    \item KM: quilômetros percorrido na viagem;
    \item Valor Minuto: valor que será usado para calcular o minuto;
    \item Minuto: tempo de viagem em minutos;
    \item Valor Mínimo: é o valor que o motorista recebe de acordo com a região.
\end{itemize}
O cálculo acima mostrará como arrecadaremos fundos com a aplicação. \\
O Quadro 6(LINK AQUI) apresentado abaixo é o cálculo dos valores arrecadados em um dia versus a quantidade dos valores arrecadados pelo motorista em 10 horas de trabalho. \\
\newpage
\begin{figure}
    \centering
    \includegraphics{exemplos/diagramas/Valor arrecadado em um dia versus quantidade de horas trabalhadas.jpeg}
    \caption{Valor arrecadado em um dia \textit{versus} quantidade de horas trabalhadas.}
    \label{fig:Valor arrecadado em um dia versus quantidade de horas trabalhadas.}
    \fonte{Os autores.}
\end{figure}\\
O Quadro 7(LINK AQUI) mostra o comparativo entre a receita mensal menos ( - ) o valor do custo mensal informado no Quadro 4(LINK AQUI).\\

\newpage
\begin{figure}
    \centering
    \includegraphics{exemplos/diagramas/Valor arrecadado mensal menos custo mensal.jpeg}
    \caption{Valor arrecadado mensal menos custo mensal.}
    \label{fig:Valor arrecadado mensal menos custo mensal.}
    \fonte{Os autores.}
\end{figure}\\

\section{Escopo do Projeto}
O escopo do projeto é definido para que a equipe consiga guiar as sprints e desenvolver a aplicação de maneira linear. Os itens descritos abaixo serão explicados mais adiante, na seção de requisitos funcionais, Apêndice E(LINK AQUI), requisitos não funcionais, Apêndice F(LINK AQUI) e histórias de usuários Apêndice D(LINK AQUI).

\subsection{Prova de Conceito}
A prova de conceito é utilizada para denominar um modelo prático que possa provar o conceito estabelecido neste projeto. Para a realização da POC, foi realizado o desenvolvimento do cadastro do usuário, pet e motorista e sua autenticação no sistema ao logar no Carrara Pets. 
\susection{Produto mínimo víavel (MVP)}
O MPV(LINK AQUI) define as funcionalidades essenciais para que o projeto funcione de maneira mais simples. Para isso, considera-se os itens desenvolvidos no POC(LINK AQUI) e mais alguns itens listados a seguir:
        \begin{itemize}
            \item Recuperar senha.
            \item Alterar dados cadastrados.
            \item Solicitação de transporte.
            \item Calcular tempo de viagem.
            \item Acompanhar viagem.
            \item Verificar forma de pagamento escolhida.
            \item Comparar saldo em conta com o valor da viagem.
            \item Cobrança de empresa externa para cartão.
            \item Finalizar viagem.
            \item Cancelar viagem.
            \item Avaliar viagem.
        \end{itemize}\\

\susection{Entrega Final}
Após entregar o POC(LINK AQUI) e MVP(LINK AQUI), buscamos finalizar a aplicação, seguindo o cronograma de planejamento criado no início do desenvolvimento desse projeto. Para a finalização da aplicação será necessário a inclusão de mais algumas funcionalidades, melhoria de algumas partes do desenvolvimento até o momento.
\begin{itemize}
    \item Comparar saldo em conta com o valor da viagem;
    \item Visualizar avaliações de outros usuários;
    \item Histórico de viagens;
    \item Indicação de parceiros;
    \item Catálogo de melhores motoristas por região;
    \item Personalização de fotos;
    \item Incluir valor adicional para o motorista;
    \item Criar tela de motoristas favoritos;
    \item Agendar viagens e serviços;
    \item Cobrança via cartão de crédito;
    \item Programa de pontos;
\end{itemize}\\
\section{Arquitetura da Aplicação}
Para a arquitetura do projeto foi definido a arquitetura em 3 camadas para melhor interação e controle. Por ser um método que proporciona 3 ambientes ou sistemas distintos, tornando assim o desenvolvimento mais ágil, escalabilidade mais simples e precisa, e menor possibilidade de indisponibilidade da aplicação e alto nível de segurança dos dados.\\
A arquitetura de três camadas é uma arquitetura de aplicativo de software bem estabelecida que organiza aplicativos em três camadas de computação lógica e física: a camada de apresentação ou interface do usuário; a camada do aplicativo, onde os dados são processados; e a camada de dados, em que os dados associados ao aplicativo são armazenados e gerenciados. (IBM, 2020)(LINK AQUI). Usando essa arquitetura teremos um método mais prático para escalar a aplicação, como também para trabalhar em partes específicas de modo que não prejudique a aplicação e a experiência do usuário.\\
\newpage
\begin{figure}
    \centering
    \includegraphics{exemplos/diagramas/Arquitetura Geral da Aplicação.jpeg}
    \caption{Arquitetura Geral da Aplicação}
    \label{fig:Arquitetura Geral da Aplicação}
    \fonte{Os autores.}
\end{figure}\\
Conforme o desenho da arquitetura geral, na parte do \textit{front-end(LINK AQUI)} da aplicação, teremos um sistema que rodará o \textit{Expo(LINK AQUI)} e \textit{React Native(LINK AQUI)}, desse modo, quando o cliente acessar o aplicativo abrirá a tela inicial e dará a opção de logar com a conta \textit{Google(LINK AQUI)} ou logar com a conta padrão. \\
Quando o cliente solicitar o acesso via Google, a aplicação irá acessar o \textit{Firebase(LINK AQUI)}, onde buscará junto ao \textit{Google API(LINK AQUI)} a validação da conta do cliente, e com isso, gerará um Token de validação que, devolverá o dado para o aplicativo, e por fim,  será enviado para o banco informando que o usuário estará se logando no aplicativo e salvando a informação. Diferente de quando o usuário decidir acessar com o endereço eletrônico e senha, o aplicativo levará os dados para o banco, validando a informação, e depois liberará  o acesso à conta do usuário.\\
Com o usuário dentro da aplicação, todos os dados processados do usuário serão enviados para o \textit{Docker(LINK AQUI)} avaliar, e depois será armazenado os dados no banco dentro do perfil do usuário.  \\
Como no início, a demanda será pequena, a equipe acompanhará esse processo para análise e melhoria, para diminuir o máximo possível a resposta e usabilidade da aplicação.\\

\subsection{Escalabilidade}
Escalabilidade de uma aplicação é a capacidade em que um sistema ou componente pode ser modificado para atender um ou mais problemas (GREGOL, 2011)(LINK AQUI). Atualmente, a escalabilidade deve estar sempre entre as prioridades (ENDEAVOR, 2015)(LINK AQUI), ou seja, a escalabilidade é uma característica desejada na aplicação.\\
Como um projeto é um aplicativo de transporte e tem potencial de aumento expressivo na quantidade de usuários e solicitações de acordo com seu crescimento. Por isso, é preciso da escalabilidade horizontal (GREGOL, 2011)(LINK AQUI), pois será necessário distribuir o processamento em diversas máquinas para que a aplicação responda rapidamente às requisições. Para atender essas solicitações, foi escolhido o PostgreSQL que possui a capacidade de expandir com facilidade sem perder a qualidade, de modo que agregam valor ao sistema. O Heroku(LINK AQUI), que será utilizado para escalar o sistema, limita seus recursos e serviços nos planos gratuitos, ou seja, no momento que for preciso escalar o sistema, precisará assinar planos para que possa consumir mais recursos dessas plataformas. Por isso, o Carrara Pets(LINK AQUI) usará a receita para arcar com essas despesas.\\
A necessidade de escalar o sistema provém para que a aplicação funcione de forma fluida e atenda as expectativas e satisfação dos usuários.

\section{Banco de Dados}
A aplicação tem como fonte de dados o Banco de Dados Relacional que, baseado no modelo, fornece acesso a pontos de dados relacionados entre si, possibilitando uma maneira intuitiva e direta de representar dados em tabelas. Neste modelo, cada linha na tabela é um registro com uma chave exclusiva, e suas colunas possuem atributos dos dados e cada registro. (ORACLE, 2022)(LINK AQUI). 

\section{Critérios de Segurança, Privacidade e Legislação}
    O aplicativo precisa gerenciar os dados e as informações pessoais do usuário de modo seguro, com o nível de permissão adequado(GOOGLE, 2022), visando a busca por definir medidas de manuseio e proteção dos dados dos usuários de qualquer que seja a aplicação. Será usada também a Lei Nº 13.853, implantada em 8 de julho de 2019, também chamada de Lei Geral de Proteção de Dados Pessoais (LGPD) (BRASIL, 2018).
    Como qualquer empresa séria que se preza, entende-se que os dados dos usuários e sua segurança são de extrema importância, com isso em mente, foram escolhidas as seguintes medidas de segurança:
\begin{itemize}
    \item \textbf{Senha com complexidade :} 
    O usuário na hora do cadastro no aplicativo irá solicitar que crie uma senha forte usando o padrão de no mínimo 8 dígitos com letras, números e pelo menos um caracter especial. Visando diminuir as chances de alguém adivinhar a senha, além de adicionar um limite de tentativas.
    \item \textbf{Verificação de duas etapas :} 
    Para validar e verificar a identidade do cliente, depois de se cadastrar no app é enviado um e-mail de validação para o usuário e um sms para verificar o telefone informado.
    \item \textbf{Criptografia de senha :} 
    Para a aplicação ser mais segura, será implantada a criptografia nos dados sensíveis dos clientes, como senhas, documentos dos clientes e dados bancários por meio de criptografia Hash.
\end{itemize}

\section{Critérios de LOG}
    Log é um arquivo de texto, XML e etc, que é gerado pela aplicação para descrever o seu funcionamento. Esse arquivo permite identificar problemas que podem estar ocorrendo dentro do sistema.
    Cada linha do Log contém uma informação a respeito de alguma alteração no software, ajudando a identificar falhas em requisições, respostas ou problemas nas funcionalidades do sistema.
    Para parte de logs será usado o Papertrail que é um software adicional do Heroku que analisa mensagens de log para detectar possíveis erros no sistema e disponibiliza notificações instantâneas automatizadas. 

\section{Processo de integração contínua}
    É uma prática de desenvolvimento de software em que os membros de uma equipe integram seu trabalho com frequência, geralmente cada pessoa integra pelo menos diariamente - levando a várias integrações por dia. Cada integração é verificada por uma compilação automatizada (incluindo teste) para detectar erros de integração o mais rápido possível (MARTIN , 2001).
    Para a melhoria contínua da aplicação, será utilizado o método de versionamento com o intuito de melhorar a aplicação criando melhoria de funcionalidades, correções de bugs, além de implementar novas opções e parcerias.

\section{Ferramentas para Testes automatizados e Análise Estática}
\textbf{Jest :} 
Para testes usaremos o Jest como principal framework para testes unitários e de unidade. O motivo pelo qual o escolhemos é principalmente pela sua simplicidade e facilidade de utilização e implementação.
\textbf{Testing Library :}
Também para os testes e para conseguir implementar o Jest para o mobile temos o Testing Library. Com muitas funcionalidades para facilitar a implementação e uma conversa muito boa com o Jest, esta será a nossa solução para testes.  
% Para facilitar a manutenção é sempre melhore criar um arquivo por capitulo, para exemplo isso não é necessário 

%---------------------------------------------------------------------------------------
\chapter{Modelo Teórico e Pressupostos (ou Hipóteses) da Pesquisa}




%---------------------------------------------------------------------------------------
\chapter{Métodos de Pesquisa}
\explicacao{Para trabalho da Pós Graduação}


\section{Tipo de Pesquisa}


\section{Plano Amostral (se Pesquisa Quantitativa)}


\section{Instrumento de Pesquisa e Escalas Utilizadas (Escalas se Pesquisa Quantitativa)}


\section{Coleta de Dados}


\section{Análise de Dados}



%---------------------------------------------------------------------------------------
\chapter{Resultados da Pesquisa}
\explicacao{Para trabalho da Pós Graduação}

\section{Discussão dos Resultados Observados}


%---------------------------------------------------------------------------------------





\chapter{Modelagem de dados}

\begin{sidewaysfigure}[htb]
    \caption{MER}
    \includegraphics[width=0.9\textwidth]{exemplos/diagramas/mer.png}
    \fonte{Autores}
\end{sidewaysfigure}

\newpage
\begin{sidewaysfigure}[htb]
    \caption{DER}
    \includegraphics[width=0.7\textwidth]{exemplos/diagramas/der.png}
    \fonte{Autores}
\end{sidewaysfigure}

\newpage
%\begin{sidewaysfigure}[htb]
    \caption{Casos de uso - Cliente}
	\includegraphics[width=0.9\textwidth]{exemplos/diagramas/Casos de uso_Cliente 1.PNG}
	\includegraphics[width=0.9\textwidth]{exemplos/diagramas/Casos de uso_Cliente 2.PNG}
	\includegraphics[width=0.9\textwidth]{exemplos/diagramas/Casos de uso_Cliente 3.PNG}
	\fonte{Autores}
%\end{sidewaysfigure}

\newpage
%\begin{sidewaysfigure}[htb]
    \caption{Casos de uso - Motorista}
	\includegraphics[width=0.8\textwidth]{exemplos/diagramas/Casos de uso_Motorista 1.PNG}
	\includegraphics[width=0.8\textwidth]{exemplos/diagramas/Casos de uso_Motorista 2.PNG}
	\fonte{Autores}
%\end{sidewaysfigure}

\newpage
\begin{sidewaysfigure}[htb]
    \caption{Diagrama de classe}
    \includegraphics[width=0.5\textwidth]{exemplos/diagramas/tcc-diagram-class.png}
    \fonte{Autores}
\end{sidewaysfigure}



% Para facilitar a manutenção é sempre melhore criar um arquivo por capitulo, para exemplo isso não é necessário 



% exemplos de escrita LaTeX e erros comuns
%\input{exemplos/exemplos}
%\input{erros/erros-comuns-documentos}
%\input{erros/erros-comuns-projetos}
%\input{erros/revisao}



% ---
% Conclusão (outro exemplo de capítulo sem numeração e presente no sumário)
% Dependendo do trabalho desenvolvido ele pode ter uma Conclusão ou Considerações finais
% Para trabalhos de disciplina utilizar Considerações Finais
% ---
\chapter{Considerações Finais}

Esse capítulo tem como intuito descrever o desenvolvimento da aplicação feita pela equipe, demonstrando as dificuldades no decorrer do projeto. Como também informando as lições aprendidas com essa experiência.

\section{Principais Dificuldades}
Desde o início do projeto, foi compreendido pelos componentes que não seria um trabalho fácil, seria necessário auto-disciplina, dedicação  e principalmente, o apoio coletivo da equipe. De maneira inicial, foi difícil, visto que a equipe encontrava-se em níveis diferentes de conhecimento e desenvolvimento. \\
 No início, a dificuldade encontrada, ocorreu quando os integrantes da equipe tinham tempo limitado para o desenvolvimento do projeto. Os integrantes trabalham em tempo integral, fazendo com que, o único tempo disponível seria de noite ou finais de semana. Mesmo reorganizando as funções, cada integrante não teria muito tempo para realizar as funções destinadas a cada um. \\
Imprevistos aconteceram durante a semana, onde interferiu diretamente no projeto. Desse jeito, dois integrantes do grupo saíram, o Kelvin e o Lucas, aumentando a carga de trabalho e a responsabilidade de todos.\\
A segunda dificuldade ocorreu quando os integrantes informaram que teriam conhecimentos básicos com as tecnologias utilizadas, como React Native, PostgreSQL, Expo, etc. Nem todos possuíam uma grande curva de aprendizado e, devido a pouca disponibilidade, isso acabou atrapalhando o desenvolvimento do projeto. Como tentativa de amenizar esse problema, realizamos algumas videochamadas para nos ajudarmos.\\


\section{Lições Aprendidas}
Ao decorrer do projeto, a equipe observou que teria que se comunicar melhor sobre o projeto, para evitar retrabalhos, pois com a falta dessa comunicação fez com que alguns tópicos fossem refeitos como, os Casos de Uso, MER, DER, etc.\\
Os Wireframes passaram a ser refeitos no decorrer do projeto, por conta das correções de funcionalidades do usuário. \\
Após identificar os pontos de melhoria, as mudanças nos padrões de comportamento trouxeram muito mais união à equipe. Observaram-se muitos pontos positivos durante o desenvolvimento, relacionados à empatia, paciência  e esforço de cada integrante durante o trabalho. Desta forma, a equipe teve uma melhora muito grande ao final do projeto. \\


\section{Funcionalidades Futuras}
Por se tratar de um aplicativo de transporte, o projeto tem potencial para ter grandes mudanças no decorrer do tempo e das necessidades que vão aparecer, de melhoria e atualizações que podem abranger leis e normas públicas, como novas funcionalidades para usuários, de modo que o App se torne mais atrativo para os usuário e futuras parcerias. Por exemplo, a inclusão de transporte de Pets usando aviões em viagens nacionais como internacionais.\\
Para o futuro, buscamos ter grande aproveitamento no crescimento do aplicativo para uso não só em território nacional como internacional, gerando assim questões como mais idiomas para o aplicativo, usar servidores externos ou até mesmo parcerias fora do Brasil.\\
Visando o cliente final, é desejado ter uma interface mais simples visando pessoas idosas ou que não possuem grande conhecimento em tecnologia usando aplicativos.\\





\input{pos-glossario.tex}
\chapter{Referências}
ABINPET. Mercado Pet Brasil 2021. Disponível em: <http://www.abinpet.org.br/download/abinpet_folder_2021.pdf> Acesso em 14 de Abril de 2022.
INSTITUTO PET BRASIL.Censo Pet: 139,3 milhões de animais de estimação no Brasil.2019.Disponível em: <http://institutopetbrasil.com/imprensa/censo-pet-1393-milhoes-de-animais-de-estimacao-no-brasil/> Acesso em: 14 de Abril de 2022.
GOVERNO FEDERAL DO BRASIL. O Brasil registrou mais de 234 milhões de acessos móveis em 2020. Agência Nacional de Telecomunicações. 2021. Disponível em: <https://www.gov.br/pt-br/noticias/transito-e-transportes/2021/05/brasil-registrou-mais-de-234-milhoes-de-acessos-moveis-em-2020> Acesso em: 14 de Abril de 2022.
HEROKU. Heroku Pricing. 2020. Disponível em: <https://www.heroku.com/pricing#app-types-header>.Acesso em: 14 abr. 2022.
RIBEIRO, A. F. de A. Cães Domesticados e os benefícios da interação. Revista Brasileira de Direito Animal, Salvador, v. 6, n. 8, 2014. DOI: 10.9771/rbda.v6i8.11062. Disponível em: https://periodicos.ufba.br/index.php/RBDA/article/view/11062. Acesso em: 14 abr. 2022.
SILVA, N. A.; MARISCO, G. A RELAÇÃO DE ANIMAIS DOMÉSTICOS NA EDUCAÇÃO E SAÚDE. Interfaces Científicas - Saúde e Ambiente, [S. l.], v. 7, n. 1, p. 71–78, 2018. DOI: 10.17564/2316-3798.2018v7n1p71-78. Disponível em: https://periodicos.set.edu.br/saude/article/view/5491. Acesso em: 14 abr. 2022.
SOUZA, A. S. de. Direitos dos animais domésticos: análise comparativa dos estatutos de proteção. Revista de Direito Econômico e Socioambiental. v. 5, n. 1, p. 110–132, 2014. Disponível em: <https://periodicos.pucpr.br/direitoeconomico/article/view/6242>. Acesso em: 14 abr. 2022.
GOOGLE PLAY CONSOLE. Disponível em: <https://play.google.com/console/u/0/signup>.Acesso em: 16 abr. 2022.
APPLE DEVELOPER PROGRAM.Isenção da taxa de assinatura no Apple Developer Program
. Disponível em: <https://play.google.com/console/u/0/signup>.Acesso em: 16 abr. 2022.
Docker. Docker Pricing. 2022. Disponível em: <https://www.docker.com/pricing/>.Acesso em: 16 abr. 2022.
GREGOL, R. E. W. Recursos de escalabilidade e alta disponibilidade para aplicações web.Repositório de Outras Coleções Abertas, p. 17–18, 2011.
ENDEAVOR. Quão longe sua ideia pode ir? Descubra avaliando a escalabilidade dela. 2015. Disponível em: <https://endeavor.org.br/estrategia-e-gestao/escalabilidade/>.Acesso em: 17 abr. 2022.
GOOGLE DEVELOPERS. Disponível em: <https://developer.android.com/docs/quality-guidelines/core-app-quality?hl=pt-br#sc>.Acesso em: 18 abr. 2022.
BRASIL. Lei Geral de Proteção de Dados (2018). 2018. Disponível em: <http://web.archive.org/web/20200915225322/http://www.planalto.gov.br/ccivil_03/_ato2015-2018/2018/lei/L13709.htm>. Acesso em: 18 abr. 2022.
 MARTIN FOWLER.Continuous Integration. 2001.Disponível em: <https://martinfowler.com/articles/continuousIntegration.html>.Acesso em: 18 abr. 2022.
PESSANHA, L.; PORTILHO, F. Comportamentos e padrões de consumo familiar em torno dos “pets”. IV ENEC - Encontro Nacional de Estudos do Consumo, 2008.
GRAF, C. T. O comportamento do consumidor no mercado pet e a relação entre os cães e as pessoas. Universidade Regional do Noroeste do Estado do Rio Grande do Sul, 2016.
CHEN, A.; HUNG, K.; PENG, N. A cluster analysis examination of pet owners ’consumption values and behavior –segmenting owners strategically. Journal of Targeting, Measurement and Analysis for Marketing, v. 20, n. 2, p. 117–132, 2012.
Silvaet.al.Petshow.2021.Disponível em:<https://svn.spo.ifsp.edu.br/svn/a6pgp/S202001-PI/HYVE/Documentos/5-EntregaFina>.Acesso em: 01 mai. 2022.
PRODEST.O uso de aplicativos na sociedade.Disponível em:
<https://prodest.es.gov.br/o-uso-de-aplicativos-na-sociedade#:~:text=Os\%20aplicativos\%20fazem\%20cada\%20vez,op\%C3\%A7\%C3\%B5es\%20de\%20lazer\%20com\%20facilidade.> Acesso em: 03 mai. 2022.
CARVALHO, J. O. F. D. O papel da interação humano-computador na inclusão digital. Transformação, V. 15, n. spe, p. 75-89. Campinas, Dezembro, 2003 .Disponível em: <http://www.scielo.br/scielo.php?script=sci_arttext&pid=S0103-37862003000500004&lng=en&nrm=iso>. Acesso em: 03 mai. 2022.
GLOBOESPORTE.COM. Uso de aplicativos de saúde deve aumentar nos
próximos anos, segundo pesquisa. Grupo Globo. Rio de Janeiro.2019. Disponível
em:<https://globoesporte.globo.com/eu-atleta/noticia/uso-de-aplicativos-de-saude-deve-aumentar-nos-proximos-anos-segundo-pesquisa.ghtml> .  Acesso em: 03 mai. 2022
BRITO, S. Quem usa mais o smartphone: Brasil ou Estados Unidos? Revista Veja digital. Editora Abril. Janeiro de 2021. Disponível em: 
<https://veja.abril.com.br/tecnologia/quem-usa-mais-o-smartphone-brasil-ou-estados-unidos/> . Acesso em: 03 mai. 2022.
SANTOS, A. Brasil, segundo país onde o mercado de aplicativos mais cresce.Terra.2020. Disponível em:<https://www.terra.com.br/noticias/dino/brasil-segundo-pais-onde-o-mercado-de-aplicativos-mais-cresce,1fd9d38aa995ad8ca1243f6c58080f79u2ee8tfj.html#:~:text=Os\%\%20realizados\%20pela\%20empresa,m\%C3\%A9dio\%20real\%20de\%2030\%20apps.&text=Houve\%20um\%20aumento\%20de\%2030,aplicativos\%20do\%20Google\%2C\%20em\%202020> .  Acesso em: 03 mai. 2022.
INSTITUTO PET BRASIL. Censo Pet: 139,3 milhões de animais de estimação no Brasil. 2019. Disponível em: <http://institutopetbrasil.com/imprensa/censo-pet-1393-milhoes-de-animais-de-estimacao-no-brasil/>. Acesso em: 19 mai. 2022.
PUGA, E. F. e. M. G. S. Banco de dados: Implementação em SQL, PL/SQL e Oracle 11g. 1. ed. São Paulo: Pearson Universidades, 2013. Acesso em: 21 mai. 2022.
IBM. Arquitetura de três camadas. 2020. Disponível em: 
<https://www.ibm.com/br-pt/cloud/learn/three-tier-architecture#toc-benefcios--jxGrdA7u>. Acesso em: 22 mai. 2022.
https://www.w3schools.com/js/js_conventions.asp

% ----------------------------------------------------------
% Apêndices
% Documentos gerados pelo próprio autor
% ----------------------------------------------------------

% ---
% Inicia os apêndices
% ---
\begin{apendicesenv}

% Imprime uma página indicando o início dos apêndices
\partapendices

% ----------------------------------------------------------
\chapter{QUESTÃO DE PESQUISA}
% ----------------------------------------------------------

\subsection{Como os animais são transportados ?}
\label{p1}
\subsection{Como solicitar uma viagem ?}
\label{p2}
\subsection{Como será a higienização do carro?}
\label{p3}
\subsection{O aplicativo irá permitir quantos animais no carro ?}
\label{p4}
\subsection{O aplicativo terá agendamentos ?}
\label{p5}
\subsection{Será possível, que o animal viaje sem o seu dono ?}
\label{p6}
\subsection{Teremos parcerias ?}
\label{p7}
\subsection{Como serão os métodos de pagamentos e como esse valor será cobrado ?}
\label{p8}
\subsection{Como será a comunicação entre o passageiro e o motorista ?}
\label{p9}
\subsection{Como será o método de contratação do motorista ?}
\label{p10}
\subsection{Como será o método de avaliação do motorista ?}
\label{p11}
\subsection{Como será o benefício comum e o premium ?}
\label{p12}
\subsection{Como será o histórico de corridas ?}
\label{p13}
\subsection{Quais animais serão permitidos ?}
\label{p14}
\subsection{Será cobrado alguma taxa extra aos passageiros, para os condutores conseguirem realizar a manutenção no carro a cada viagem ?}
\label{p15}
\subsection{Como será definida a rota de transporte ?}
\label{p16}
\subsection{Como será o chat durante a corrida ?}
\label{p17}
\subsection{Como funcionará o sistema de pontos fidelidade ?}
\label{p18}
\subsection{Como vai funcionar a compra de descontos (Sistemas de pontos)  ?}\label{p19}

\section{Respostas }\\
\subsection{Resposta da pergunta \ref{p1}}
No Carrara Pets, para garantir a segurança e o conforto aos animais, cães usam um cinto de segurança especial que prende no gancho da coleira, e os gatos são transportados dentro de uma caixa específica. Os veículos são higienizados ao final de cada viagem. Caso o pet realize a viagem sozinho, a plataforma envia um SMS automático ao usuário que a corrida foi finalizada, mas também é possível acompanhar o percurso em tempo real por meio do aplicativo. Com o aplicativo, é possível acompanhar a viagem em tempo real, e o pagamento é feito pelo aplicativo.

\subsection{Resposta da pergunta \ref{p2}}
Para solicitar uma corrida, basta abrir o aplicativo e informar os locais de encontro e de destino. O usuário pode acompanhar o trajeto em tempo real, e também consegue realizar o agendamento de corridas. O pagamento pode ser feito com cartão de crédito, a um preço fixo. Com o objetivo de garantir um bom atendimento, é possível visualizar a avaliação dos motoristas, feita por outros usuários da plataforma.

\subsection{Resposta da pergunta \ref{p3}}
Todos os veículos estarão equipados com Kit de proteção que inclui guia, cinto peitoral e focinheira para cães. No caso dos gatos, conectamos a sua caixa de transporte própria ao nosso sistema de segurança. A higiene do veículo também é bastante cuidadosa. Os bancos possuem uma capa protetora de assento, que são aspiradas ao final de cada corrida e recebem uma solução fungicida (combate fungos), viricida (inibe a proliferação de um vírus num processo infeccioso) e bactericida (antibióticos que destroem bactérias) de uso veterinário. É um processo muito importante, pois evita disseminação de doenças entre os animais e humanos também, além de deixar o carro limpo para a próxima corrida.

\subsection{Resposta da pergunta \ref{p4}}
No aplicativo iremos possuir um campo específico para informar a quantidade de pets que irão viajar no veículo e os portes dos animais. 
\textbf{Regras:} 
\begin{itemize}
    \item 1 animal - porte grande, médio ou pequeno.
    \item 2 animais - porte grande e um médio 
    \item 3 animais - porte médio e pequeno 
    \item 4 animais - porte pequeno.
\end{itemize} 

\subsection{Resposta da pergunta \ref{p5}}
Sim. Você contará com a comodidade de agendamento prévio de corridas. Funciona do seguinte modo: você seleciona o dia, a hora e o local para o nosso motorista ir buscar o seu pet e você no local solicitado.

\subsection{Resposta da pergunta \ref{p6}}
Sim. Você poderá realizar o despacho do seu pet desacompanhado. Você não vai precisar acompanhá-lo na viagem se não puder ou quiser, basta ter um responsável para entregar o animal ao motorista no local de origem e outro para receber no local de destino. 
\textit{Pontos Negativos com o serviço de transportes de passageiros ou públicos para deslocar com o seu pet:}
\begin{enumerate}
    \item Insegurança - a falta de equipamentos de segurança para prender corretamente o seu pet ao cinto de segurança e a falta de conhecimento do condutos podem causar acidentes.
    \item Desrespeito às normas de trânsito - O código de trânsito brasileiro (CTB) estabelece normas para o transporte dos animais, motoristas que não possuem treinamentos específicos, podem desconhecê-las e acabam transportando o seu cão de forma errada. 
    \item Motorista despreparados - o motorista não tem experiência e nem treinamento para fazer o transporte dos pet. Além disso, podem ficar irritados se o animal fizer sujeira no carro. 
    \item Falta de equipamentos - os veículos comuns não possuem os acessórios de segurança e higiene para pet.
\end{enumerate}

\subsection{Resposta da pergunta \ref{p7}}
Sim. Iremos ter parcerias com estabelecimentos de pet, como Pet Shop, veterinário, Banho e Tosa. Caso a pessoa seja dono ou gestor de um estabelecimento, poderá entrar em contato conosco e pedir os nossos serviços de transporte de animais domésticos à sua empresa.

\subsection{Resposta da pergunta \ref{p8}}
O app permite cadastrar métodos como cartão de crédito ou dinheiro, para que o passageiro possa escolher a melhor maneira de pagar suas corridas.

\subsection{Resposta da pergunta \ref{p9}}
O método de comunicação entre o passageiro e o motorista será através de um chat, onde os dois indivíduos podem se comunicar entre possíveis atrasos, localizações, consultas de veterinários, comportamento do animal.

\subsection{Resposta da pergunta \ref{p10}}
A contratação de um motorista parceiro, será realizada através do próprio aplicativo em uma aba específica para motoristas parceiros “Seja Parceiro”. O processo de cadastro será feito em três etapas, a primeira que seria o pré-cadastro, será preenchido as informações do motorista como Nome, CPF, CNH, telefone, e-mail, idade, endereço, placa e modelo do veículo, se tem pet ou não e uma carta respondendo a seguinte pergunta “Por que deverá ser aceito ?”. Na segunda etapa teremos uma entrevista com um psicólogo e um teste psicanalítico e psicotécnico, e por fim a etapa de confirmação de cadastro, onde o candidato irá receber um email com o resultado.

\subsection{Resposta da pergunta \ref{p11}}
A avaliação do motorista será solicitada ao final de toda corrida, no app do nosso cliente, em formato de pop-up no seguinte modelo : 
\\
    \includegraphics{exemplos/diagramas/star.PNG}
\\
Onde irá de uma “patinha” a cinco “patinhas”.
Obs: Avaliações de uma “patinhas” a duas “patinhas”, após a avaliação será apresentado um campo de texto para incluir observação de um possível problema ou feedback.

\subsection{Resposta da pergunta \ref{p12}}
\textbf{Benefício Comum:}
\begin{itemize}
    \item Apenas transportes.
\end{itemize} 
\textbf{Benefícios para o Premium:}
\begin{itemize}
    \item Aumento no ganho de pontos.
    \item Descontos adquiridos por meio do sistema de pontos, serão dobrados.
    \item Prioridade em motoristas bem avaliados.
\end{itemize}

\subsection{Resposta da pergunta \ref{p13}}
O histórico de corridas será apresentado no ícone “
    \includegraphics[scale = 0.65]{{exemplos/diagramas/hist.PNG}}
” dentro do aplicativo, e será  composto pelas vinte últimas corridas do cliente. Podendo ter acesso às seguintes informações das corridas : Local de partida, local de destino, rota percorrida, valor da corrida, pet transportado, nome do motorista, modelo e placa do veículo, gastos adicionais (caso tiver), método de pagamento e observações da corrida.

\subsection{Resposta da pergunta \ref{p14}}
Vamos ter uma regra bastante rígida em relação aos pets que poderão ser transportados, onde aceitaremos o cadastro e transporte de somente animais domésticos. Quaisquer tentativas de transporte de animais silvestres serão repudiadas com sujeição a punições.

\subsection{Resposta da pergunta \ref{p15}}
\textbf{Caso tenha a taxa:} 
O nosso aplicativo deixará claro que animais a serviço - como cães-guia, por exemplo - estão isentos de cobrança da taxa, como estipula leis estaduais e federais.

\subsection{Resposta da pergunta \ref{p16}}
O processo de definição será por meio de grafos e informações de trânsito ao vivo, utilizando o ponto de partida e traçando a rota até o ponto de destino, podendo adicionar paradas adicionais.

\subsection{Resposta da pergunta \ref{p17}}
O aplicativo irá contar com um chat durante todas as corridas, onde o motorista e o usuário poderá trocar informações sobre possíveis consultas e detalhes do passeio com o pet, também será possível adicionar documentos (Comprovantes/Diagnósticos) ao chat.

\subsection{Resposta da pergunta \ref{p18}}
Teremos como métodos de fidelidade com o usuário o sistema de pontos, onde ao final de todas as corridas nossos usuários receberão pontos conforme a distância da corrida. O cálculo dos pontos será feito da seguinte forma (distância da viagem em metros) / 10, já para usuários premium o cálculo será ((distância da viagem em metros) / 10) + ((distância da viagem em metros) * 0.04) 
Ex (usuário comum): Viagem de 2,6 km, o usuário receberá 260 pontos.
Ex (usuário premium): Viagem de 2,6km, o usuário receberá 364 pontos.
Esses pontos poderão ser trocados em descontos para viagens futuras.

\subsection{Resposta da pergunta \ref{p19}}
Os valores dos descontos serão :\\
5\% de desconto - 600 pontos.\\
10\% de desconto - 1000 pontos.\\
30\% de desconto - 2800 pontos.\\
50\% de desconto - 4500 pontos.

% ----------------------------------------------------------
\chapter{SPRINTS}
% ----------------------------------------------------------
Essa seção contém o conteúdo desenvolvido durante todas as nossas Sprint(LINK AQUI), separados por onze períodos de uma semana cada reunião. Cara Sprint teve em média duas horas, visando respeitar a disponibilidade dos integrantes do grupo.\\

\begin{figure}
    \centering
    \includegraphics{exemplos/diagramas/Sprints de 11 semanas realizada pela equipe..jpeg}
    \caption{Sprints de 11 semanas realizada pela equipe.}
    \label{fig:Sprints de 11 semanas realizada pela equipe.}
    \fonte {Os Autores}
\end{figure}
% ----------------------------------------------------------
%WIREFRAME
\chapter{Wireframe}
\begin{center}
\includegraphics[width=230]{exemplos/Wireframe/Wireframe1.PNG}
\includegraphics[width=220]{exemplos/Wireframe/Wireframe2.PNG}
\\
\includegraphics[width=250]{exemplos/Wireframe/Wireframe3.PNG}
\includegraphics[width=100]{exemplos/Wireframe/Wireframe4.PNG}
\\
\includegraphics[width=250]{exemplos/Wireframe/Wireframe5.PNG}
\\
\includegraphics[width=250]{exemplos/Wireframe/Wireframe6.PNG}
\\
\includegraphics[width=250]{exemplos/Wireframe/Wireframe7.PNG}
\includegraphics[width=100]{exemplos/Wireframe/Wireframe7_1.PNG}
\\
\includegraphics[width=250]{exemplos/Wireframe/Wireframe8.PNG}
\\
\includegraphics[width=200]{exemplos/Wireframe/Wireframe9.PNG}
\includegraphics[width=200]{exemplos/Wireframe/Wireframe10.PNG}
\includegraphics[width=200]{exemplos/Wireframe/Wireframe11.PNG}
\\
\includegraphics[width=200]{exemplos/Wireframe/Wireframe12.PNG}
\includegraphics[width=80]{exemplos/Wireframe/Wireframe13.PNG}
\includegraphics[width=80]{exemplos/Wireframe/Wireframe14.PNG}
\\
\includegraphics[width=80]{exemplos/Wireframe/Wireframe15.PNG}
\end{center}

% ----------------------------------------------------------
\chapter{HISTÓRIA DE USUÁRIO }
% ----------------------------------------------------------


%REQUISITOS 
\chapter{Requisitos funcionais e não funcionais}

\section{Requisitos funcionais}
\begin{quadro}[thb]
\ABNTEXfontereduzida
\caption{Requisitos Funcionais}
\label{quadro-poluido-limpo-desalinhado}
\begin{tabular}{|l|p{2cm}|l|l|l|p{6cm}|}
\hline
\thead{RF}&\thead{Requisitos\\Funcionais} & \thead{Essencial} & \thead{Importante} & \thead{Desejável} & \thead{Descrição}\\
\hline
RF001&Permitir cadastro de usuários&\circlemark& & & O sistema tem que conceder o cadastro do usuário;\\
\hline
RF002&Permitir cadastro de usuários com conta Google&\circlemark& & & A aplicação deverá permitir que o usuário se cadastre usando a conta Google.\\
\hline
RF003&Permitir cadastro dos Pets&\circlemark& & & O sistema carece que o usuário realize o cadastro do pet (cachorro ou gato).\\
\hline
RF004&Permitir o cadastro do motorista&\circlemark& & & O app deverá aceitar o cadastro do motorista;\\
\hline
RF005&Validar e-mail&\circlemark& & & A aplicação precisa validar se o e-mail informado está correto, enviando um e-mail para o tipo usuário que está solicitando o cadastro. No máximo em 10 segundos.\\
\hline
RF006&Validar celular&\circlemark& & & A aplicação deve enviar um sms validando o número do celular do cliente cadastrado e do motorista no máximo em 10 segundos.\\
\hline
RF007&Realizar o login &\circlemark& & & O sistema deverá permitir o usuário efetuar login;\\
\hline
RF008&Escolher modalidade&\circlemark& & & Quando o cliente acessar o aplicativo o cliente escolher a modalidade comum ou premium, sendo a modalidade comum somente transporte de levar até um local específico e a premium disponibiliza serviços como: transporte e cuidado em parque, acompanhamento no veterinário e etc.\\
\hline
RF009&Quantidade de pets&\circlemark& & & A aplicação deverá solicitar que o cliente informe a quantidade de pets que irá ser transportado;\\
\hline
RF010&Escolher forma de pagamento&\circlemark& & & O cliente define a forma de pagamento como: cartão de crédito com cobrança na hora ou dinheiro depositado previamente na conta.\\
\hline
\end{tabular}
\fonte{Os autores.}
\end{quadro}

\newpage
\begin{quadro}[thb]
\ABNTEXfontereduzida
\begin{tabular}{|l|p{2cm}|l|l|l|p{6cm}|}
\hline
\thead{RF}&\thead{Requisitos\\Funcionais} & \thead{Essencial} & \thead{Importante} & \thead{Desejável} & \thead{Descrição}\\
\hline
RF011&Calcular viagem&\circlemark& & & A aplicação deverá informar ao cliente o valor da viagem antes do mesmo iniciar a viagem, levando em consideração a quantidade de pets, distância e modalidade (serviços da modalidade premium);\\
\hline
RF012&Informar tempo estimado da chegada do veículo&\circlemark& & & Através de um cálculo utilizando distância entre o passageiro e o motorista, o sistema deverá estimar um tempo de chegada. \\
\hline
RF013&Solicitar Transporte &\circlemark& & & O sistema deverá possibilitar a um usuário com perfil passageiro informar um local de origem e destino, para posteriormente fazer uma solicitação de transporte. \\
\hline
RF014&Acompanhar o pet no mapa&\circlemark& & & Tanto os passageiros como motoristas deverão poder acompanhar a posição do usuário que estiverem envolvidos na viagem.\\
\hline
RF015&Conversar com o motorista&\circlemark& & & A aplicação precisa disponibilizar  um chat para a comunicação entre o cliente e o motorista;\\
\hline
RF016&Agendar viagem& &\circlemark& & Descrição: A aplicação deverá disponibilizar a disponibilidade do motorista para o cliente, caso o mesmo deseje agendar uma nova viagem previamente.\\
\hline
RF017&Permitir confirmação da viagem&\circlemark& & & Toda viagem necessita ser confirmada tanto pelos passageiros como motoristas. \\
\hline
RF018&Finalizar viagem&\circlemark& & & A aplicação só liberará a opção de finalizar a viagem ao motorista a 1 metro do final do trajeto;\\
\hline
RF019&Permitir cancelamento da viagem&\circlemark& & & O cancelamento só será permitido caso houver um incidente pelo motorista. O sistema mostrará uma tela informando os possíveis motivos para o cancelamento.\\
\hline
RF20&Realizar avaliação da viagem&\circlemark& & & Ao término de cada viagem o passageiro irá receber uma mensagem no celular para estar avaliando a experiência da viagem e do serviço feito pelo motorista. \\
\hline
\end{tabular}
\fonte{Os autores.}
\end{quadro}
\\

%-----------------------------------------------%
\newpage
\\
\section{Requisitos não funcionais}
\begin{quadro}[thb]
\ABNTEXfontereduzida
\caption{Requisitos não funcionais}
\label{quadro-poluido-limpo-desalinhado}
\begin{tabular}{|l|p{2cm}|l|l|l|p{6cm}|}
\hline
\thead{RNF}&\thead{Requisitos\\Funcionais} & \thead{Essencial} & \thead{Importante} & \thead{Desejável} & \thead{Descrição}\\
\hline
RNF001&Interface com boa usabilidade&\circlemark& & & A aplicação deve ser intuitiva, de modo que possa ser de fácil entendimento  a todas as pessoas que acessarem o aplicativo.\\
\hline
RNF002&Resposta rápida do servidor&\circlemark& & & O software tem que fazer as consultas, histórico de viagens e a autenticação em menos de 1 segundo no lado do servidor.\\
\hline
RNF003&Verificar senha&\circlemark& & & O aplicativo necessita validar se a senha é composta por letras,caracteres especiais e número.\\
\hline
RNF004&Viagem acompanhada com o dono& &\circlemark& & O cliente deve informar via aplicativo para o motorista se irá acompanhar o seu pet. (haverá um botão no app) .\\
\hline
RNF005&Exibir senha& &\circlemark& & O aplicativo deve disponibilizar a opção de visualizar a senha.\\
\hline
RNF006&Disponibilidade do aplicativo&\circlemark& & & A aplicação deve estar disponível 99,99\% do tempo para os usuários, sete dias por semana e 24 horas por dia.\\
\hline
RNF007&Restringir a quantidade de  pets&\circlemark& & & O sistema deve informar a quantidade máxima de pets que o motorista pode transportar, tendo o máximo de 4 pets. \\
\hline
\end{tabular}
\fonte{Os autores.}
\end{quadro}


% ----------------------------------------------------------
\chapter{REGRA DE NEGÓCIOS}
% ----------------------------------------------------------


\end{apendicesenv}
% ---





\input{pos-anexos}


%---------------------------------------------------------------------
% INDICE REMISSIVO - Quando necessário 
% As palavras indexadas devem ser definidas com \index{} no texto
%---------------------------------------------------------------------
\phantompart
\printindex

%---------------------------------------------------------------------

\end{document}