% ---
% Capitulo de revisão de literatura
% ---
\chapter{Revisão da Literatura}

Na antiguidade, os seres humanos e outros animais caçavam devido a seu instinto, porém em algum momento, os seres humanos entenderão que ao domesticar esses animais poderiam ter maior efetividade no seu dia a dia. No decorrer do tempo, o ser humano descobriu que ao coexistir com animais, eles os tornavam mais fortes e felizes, foram inseridos cada vez mais em seu cotidiano até se tornarem parte da família, como seres de estimação, companheiro fiel e um apoio emocional (PESSANHA; PORTILHO, 2008)(LINK AQUI).
Olhando a sociedade moderna, onde esses animais se tornaram extremamente valorizados, de modo que chegam a substituir filhos. Muitas pessoas desistem de ter filhos para ter um animal tratado como tal (PESSANHA; PORTILHO, 2008)(LINK AQUI).

\section{Aplicativos}
Com o desenvolvimento da tecnologia o uso de aparelhos celulares e seus aplicativos têm se tornado essenciais no cotidiano das pessoas, por inúmeros motivos como facilidade para acessar contas bancárias, solicitar transporte, pedir uma refeição etc. Tudo isso de maneira rápida e extremamente acessível. Segundo pesquisas, o Brasil possui um grande potencial no âmbito de dispositivos móveis, cerca de 29\% da população utiliza aplicativos para atividades corriqueiras. Outra pesquisa mostra que as pessoas do Brasil ficam em torno de 3 horas em frente ao celular diariamente, superando países como EUA. (CARVALHO, 2003; GLOBOESPORTE.COM, 2019; SANTOS, 2020; BRITO, 2021)(LINK AQUI).\\
No projeto, a ideia de criar um aplicativo ao invés de um WebApp ou uma aplicação Web, deu-se o intuito de ter maior credibilidade e funcionalidade na aplicação, além de usar uma baixa quantidade de dados móveis para o cliente final.

\section{Público alvo}
Silva (2021) apud Graf (2016) Diz que os animais são atualmente tratados como membros da família. Segundo Silva et.al. (2021), a relação entre seres domesticados e seus donos pode ser separada em:
\begin{itemize}
    \item \underline{" Afeto: O dono usa mais serviços/produtos de alta qualidade, como enfeites, xampu\\, roupas etc.;" }
    \item \underline{"Interação: Onde o dono contrata serviços de adestramento para adequar o animal \\ao seu estilo de vida e produtos para o bem-estar dele;"}
    \item \underline{"Substituição humana: O consumidor substitui as relações humanas, como filhos ou \\amigos, pelo animal de estimação, pagando por atividades de tratamentos \\veterinários de luxo, adestramento e atividades geralmente associadas a relações humanas,\\ como funerais.”}
\end{itemize}

% ---

% ---

% ---
